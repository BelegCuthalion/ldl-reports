\documentclass[a4paper, 12pt]{paper}
\usepackage{amsmath}
\usepackage{amssymb}
\usepackage{mathabx}
\usepackage{titlesec}
\usepackage{fullpage}
\usepackage{tikz-cd}
\usepackage{rotating}
\usepackage{pdflscape}
\usepackage{multicol}
\usepackage{multirow}
\usepackage{diagbox}
\usepackage[left=.5in,right=.5in,top=.5in,bottom=.5in]{geometry}
\usepackage{enumitem}
\usepackage[colorlinks, hypertexnames=false]{hyperref}
\usepackage{bussproofs}
\usepackage[normalem]{ulem}
\usepackage{amsthm}

\theoremstyle{plain}
\newtheorem{thm}{Theorem}[section]
\renewcommand{\thethm}{\arabic{thm}}
\newtheorem{lem}[thm]{Lemma}
\newtheorem{cor}[thm]{Corollary}
\theoremstyle{definition}
\newtheorem{dfn}[thm]{Definition}
\newtheorem{exam}[thm]{Example}
\newtheorem{rem}[thm]{Remark}
\newtheorem{nota}[thm]{Notation}
\newtheorem{obs}[thm]{Observation}

\setitemize{topsep=3pt,parsep=5pt,partopsep=0pt,label=,leftmargin=1.3pc}
\titleformat{\section}[runin]{\normalfont\bfseries}{\thesection}{0.5em}{}
\titlespacing{\section}{0pc}{5ex plus .1ex minus .2ex}{1pc}
\titleformat{\subsection}[runin]{\normalfont\bfseries}{\thesubsection}{0.7em}{}
\titlespacing{\subsection}{0pc}{2ex plus .1ex minus .2ex}{1pc}
\titleformat{\subsubsection}[runin]{\normalfont\bfseries}{\thesubsubsection}{0.7em}{}
\titlespacing{\subsubsection}{0pc}{2ex plus .1ex minus .2ex}{1pc}
\newcommand\eqn{\refstepcounter{equation}\tag{\theequation}}
\binoppenalty=\maxdimen
\relpenalty=\maxdimen
\newcommand{\ul}{\ulcorner}
\newcommand{\ur}{\urcorner}
\newcommand{\val}[1]{\ulcorner len1 \urcorner}
\newcommand{\caseref}[1]{\hyperref[#1]{\ref{#1}}}
\newcommand{\rot}{\rotatebox{90}}
\newcommand{\p}{\partial}
\newcommand{\todo}[1]{{\color{red}\textbf{TODO} #1}}
\newcommand{\red}{\color{red}}
\newcommand{\stl}{\mathbf{STL}}
\newcommand{\st}{\text{ST}}
\newcommand{\qst}{\mathbf{QST}}
\newcommand{\gstl}{\mathbf{GSTL}}
\newcommand{\gst}{\text{GST}}
\newcommand{\istl}{\mathbf{iSTL}}
\newcommand{\ist}{\text{iST}}
\newcommand{\igstl}{\mathbf{iGSTL}}
\newcommand{\igst}{\text{iGST}}
\newcommand{\sstl}{\mathbb{S}\mathbf{TL}}
\newcommand{\gsstl}{\mathbf{G}\mathbb{S}\mathbf{TL}}
\newcommand{\stt}{\mathbf{ST3}}
\newcommand{\qstt}{\mathbf{QST3}}
\newcommand{\gstt}{\text{GST3}}
\newcommand{\gsttl}{\mathbf{GST3L}}
\newcommand{\istt}{\text{iST3}}
\newcommand{\igstt}{\text{iGST3}}
\newcommand{\ldl}{\mathbf{LDL}}
\newcommand{\qldl}{\mathbf{QLDL}}
\newcommand{\D}{\mathcal{D}}
\newcommand{\IH}{\text{IH}}
% \newcommand{\qed}{{\begin{flushright}$\Box$\end{flushright}}}
\newtheorem{proposition}{}[section]
\EnableBpAbbreviations

\def\L{\mathcal{L}}
\def\ldl{\mathsf{LDL}}
\def\LDL{\mathbf{LDL}}
\def\iSTLN{\mathbf{iSTL(N)}}
\def\GSTN{\mathbf{GSTN}}

\def\D{\mathcal{D}}
\def\IH{\mathrm{IH}}

\begin{document}
{\noindent
	v 1 \\
{\large\textbf{A Tableau System for STL}}
}
\\
\setcounter{section}{-1}
\section{Notation} In the following $\Gamma$, $\Sigma$, $\Pi$ and $\Delta$ are names for finite multi-sets of formulas, $\delta$ for a sub-singleton of some formula, $A$, $B$ and $C$ for formulas, $p$ for propositional variables and $n$, $l$, $r$ and $k$ for natural numbers. We denote by $P$ the set of all propositional variables, and by $P(A)$ the set of those that occur in the formula $A$.

We will write $A$ for the singleton $\{A\}$ where ever it is inferable from the context.
``$,$'' is the multi-set union.
We will also write $A^1$ for $\{A\}$, write $A^{n+1}$ for $A^n, A$, write $\Gamma^n$ for $\{ A^n : A \in \Gamma \}$ and write $P(\Gamma)$ for $\bigcup_{A \in \Gamma} P(A)$

A sequent $\Gamma \Rightarrow \Delta$ is a binary relation between $\Gamma$, a multi-set of formulas, and $\Delta$, a sub-singleton of some formula.

If $A$ is a formula, then $T A$ and $F A$ are signed formulas. In this document, $\Psi$ and $\Phi$ are names for finite sets of signed formulas. We also occasionally use the name $X$ for the letters $T$ and $F$. Also notice that wherever we use the set-comprehension notation for formulas, we are constructing a multiset, and wherever we use that notation for signed formulas, it is a set. We will also use the following short-hands, where $\Gamma$ is a multiset of formulas and $\Phi$ is a set of signed formulas and $X$ is either $T$ or $F$.

$\nabla \Gamma = \{ \nabla A \mid A \in \Gamma \}$

$X \Gamma = \{ X A \mid A \in \Gamma \}$

$\Phi_X = \{ X A \mid X A \in \Phi \}$

$X_\Phi = \{ A \mid X A \in \Phi \}$

$\nabla \Phi_X = \{ X \nabla A \mid X A \in \Phi \}$



\section{Asymmetric system for STL} \quad \\

 \begin{multicols}{3}
   \begin{prooftree}
     \AXC{}
     \RightLabel{$Id$}
     \UIC{$ A \Rightarrow A$}
   \end{prooftree}
   \columnbreak
   \begin{prooftree}
     \AXC{}
     \RightLabel{$Ta$}
     \UIC{$ \Rightarrow \top$}
   \end{prooftree}
   \columnbreak
   \begin{prooftree}
     \AXC{}
     \RightLabel{$Ex$}
     \UIC{$ \bot \Rightarrow $}		
   \end{prooftree}
 \end{multicols}
 
 \begin{multicols}{3}
   \begin{prooftree}
     \AXC{$ \Gamma, A \Rightarrow \delta$}
     \RightLabel{$L \wedge_1$}
     \UIC{$ \Gamma, A \wedge B \Rightarrow \delta$}		
   \end{prooftree}
   \columnbreak
   \begin{prooftree}
     \AXC{$ \Gamma, B \Rightarrow \delta$}
     \RightLabel{$L \wedge_2$}
     \UIC{$\Gamma, A \wedge B \Rightarrow \delta$}		
   \end{prooftree}
   \columnbreak
   \begin{prooftree}
     \AXC{$\Gamma \Rightarrow A$}
     \AXC{$\Gamma \Rightarrow B$}
     \RightLabel{$R \wedge$}
     \BIC{$ \Gamma \Rightarrow A \wedge B$}		
   \end{prooftree}
 \end{multicols}
 
 \begin{multicols}{3}
   \begin{prooftree}
     \AXC{$ \Gamma, A \Rightarrow \delta$}
     \AXC{$\Gamma, B \Rightarrow \delta$}
     \RightLabel{$L \vee$}
     \BIC{$ \Gamma, A \vee B \Rightarrow \delta$}		
   \end{prooftree}
   \columnbreak
   \begin{prooftree}
     \AXC{$\Gamma \Rightarrow A$}
     \RightLabel{$R \vee_1$}
     \UIC{$\Gamma \Rightarrow A \vee B$}		
   \end{prooftree}
   \columnbreak
   \begin{prooftree}
     \AXC{$\Gamma \Rightarrow B$}
     \RightLabel{$R \vee_2$}
     \UIC{$\Gamma \Rightarrow A \vee B$}		
   \end{prooftree}
 \end{multicols}
 
  \begin{multicols}{2}
   \begin{prooftree}
     \AXC{$\Gamma \Rightarrow A$}
     \AXC{$\Gamma, B \Rightarrow \delta$}
     \RightLabel{$L \rightarrow$}
     \BIC{$\Gamma, \nabla (A \rightarrow B) \Rightarrow \delta$}		
   \end{prooftree}
   \columnbreak
   \begin{prooftree}
     \AXC{$\nabla \Gamma, A \Rightarrow B$}
     \RightLabel{$R \rightarrow$}
     \UIC{$\Gamma \Rightarrow A \rightarrow B$}		
   \end{prooftree}
 \end{multicols}
 
 \begin{multicols}{2}
  \begin{prooftree}
    \AXC{$ \Gamma \Rightarrow \delta$}
    \RightLabel{$L w$}
    \UIC{$ \Gamma, A \Rightarrow \delta$}
  \end{prooftree}
  \columnbreak
  \begin{prooftree}
    \AXC{$ \Gamma \Rightarrow$}
    \RightLabel{$R w$}
     \UIC{$\Gamma \Rightarrow A$}		
  \end{prooftree}
\end{multicols}

\begin{multicols}{1}
  \begin{prooftree}
    \AXC{$ \Gamma, A, A \Rightarrow \delta$}
    \RightLabel{$Lc$}
    \UIC{$\Gamma, A \Rightarrow \delta$}		
  \end{prooftree}
\end{multicols}

 \begin{prooftree}
   \AXC{$\Gamma \Rightarrow \delta$}
   \RightLabel{$N$}
   \UIC{$\nabla \Gamma \Rightarrow \nabla \delta$}
 \end{prooftree}

 \begin{center}
  \begin{prooftree}
    \AXC{$ \Gamma \Rightarrow A$}
    \AXC{$\Sigma, A \Rightarrow \delta$}
    \RightLabel{$cut$}
    \BIC{$\Gamma, \Sigma \Rightarrow \delta$}
  \end{prooftree}
\end{center}

By $\stl \vdash \Gamma \Rightarrow \delta$ we mean that the sequent $\Gamma \Rightarrow \delta$ provable in this system.
\pagebreak

\section{Symmetric system for STL} \quad \\

 \begin{multicols}{3}
   \begin{prooftree}
     \AXC{}
     \RightLabel{$Id$}
     \UIC{$ A \Rightarrow A$}
   \end{prooftree}
   \columnbreak
   \begin{prooftree}
     \AXC{}
     \RightLabel{$Ta$}
     \UIC{$ \Rightarrow \top$}
   \end{prooftree}
   \columnbreak
   \begin{prooftree}
     \AXC{}
     \RightLabel{$Ex$}
     \UIC{$ \bot \Rightarrow $}		
   \end{prooftree}
 \end{multicols}
 
 \begin{multicols}{3}
   \begin{prooftree}
     \AXC{$ \Gamma, A \Rightarrow \Delta$}
     \RightLabel{$L \wedge_1$}
     \UIC{$ \Gamma, A \wedge B \Rightarrow \Delta$}		
   \end{prooftree}
   \columnbreak
   \begin{prooftree}
     \AXC{$ \Gamma, B \Rightarrow \Delta$}
     \RightLabel{$L \wedge_2$}
     \UIC{$\Gamma, A \wedge B \Rightarrow \Delta$}		
   \end{prooftree}
   \columnbreak
   \begin{prooftree}
     \AXC{$\Gamma \Rightarrow A, \Delta$}
     \AXC{$\Gamma \Rightarrow B, \Delta$}
     \RightLabel{$R \wedge$}
     \BIC{$ \Gamma \Rightarrow A \wedge B, \Delta$}		
   \end{prooftree}
 \end{multicols}
 
 \begin{multicols}{3}
   \begin{prooftree}
     \AXC{$ \Gamma, A \Rightarrow \Delta$}
     \AXC{$\Gamma, B \Rightarrow \Delta$}
     \RightLabel{$L \vee$}
     \BIC{$ \Gamma, A \vee B \Rightarrow \Delta$}		
   \end{prooftree}
   \columnbreak
   \begin{prooftree}
     \AXC{$\Gamma \Rightarrow A, \Delta$}
     \RightLabel{$R \vee_1$}
     \UIC{$\Gamma \Rightarrow A \vee B, \Delta$}		
   \end{prooftree}
   \columnbreak
   \begin{prooftree}
     \AXC{$\Gamma \Rightarrow B, \Delta$}
     \RightLabel{$R \vee_2$}
     \UIC{$\Gamma \Rightarrow A \vee B, \Delta$}		
   \end{prooftree}
 \end{multicols}
 
  \begin{multicols}{2}
   \begin{prooftree}
     \AXC{$\Gamma \Rightarrow A$}
     \AXC{$\Gamma, B \Rightarrow \Delta$}
     \RightLabel{$L \rightarrow$}
     \BIC{$\Gamma, \nabla (A \rightarrow B) \Rightarrow \Delta$}		
   \end{prooftree}
   \columnbreak
   \begin{prooftree}
     \AXC{$\nabla \Gamma, A \Rightarrow B$}
     \RightLabel{$R \rightarrow$}
     \UIC{$\Gamma \Rightarrow A \rightarrow B$}		
   \end{prooftree}
 \end{multicols}
 
 \begin{multicols}{2}
  \begin{prooftree}
    \AXC{$ \Gamma \Rightarrow \Delta$}
    \RightLabel{$L w$}
    \UIC{$ \Gamma, A \Rightarrow \Delta$}
  \end{prooftree}
  \columnbreak
  \begin{prooftree}
    \AXC{$ \Gamma \Rightarrow \Delta$}
    \RightLabel{$R w$}
     \UIC{$\Gamma \Rightarrow A, \Delta$}		
  \end{prooftree}
\end{multicols}

\begin{multicols}{2}
  \begin{prooftree}
    \AXC{$ \Gamma, A, A \Rightarrow \Delta$}
    \RightLabel{$Lc$}
    \UIC{$\Gamma, A \Rightarrow \Delta$}		
  \end{prooftree}
  \columnbreak
  \begin{prooftree}
    \AXC{$ \Gamma \Rightarrow A, A, \Delta$}
    \RightLabel{$Rc$}
    \UIC{$\Gamma \Rightarrow A, \Delta$}		
  \end{prooftree}
\end{multicols}

 \begin{prooftree}
   \AXC{$\Gamma \Rightarrow \Delta$}
   \RightLabel{$N$}
   \UIC{$\nabla \Gamma \Rightarrow \nabla \Delta$}
 \end{prooftree}

 \begin{center}
  \begin{prooftree}
    \AXC{$ \Gamma \Rightarrow A, \Pi$}
    \AXC{$\Sigma, A \Rightarrow \Delta$}
    \RightLabel{$cut$}
    \BIC{$\Gamma, \Sigma \Rightarrow \Pi, \Delta$}
  \end{prooftree}
\end{center}


By $\sstl \vdash \Gamma \Rightarrow \Delta$ we mean that the sequent $\Gamma \Rightarrow \Delta$ provable in this system.

\section{Theorem} \textit{Symmetric and asymmetric systems are equivalent:} 

\quad $\sstl \vdash \Gamma \Rightarrow \Delta$ \quad iff \quad $\stl \vdash \Gamma \Rightarrow \bigvee \Delta$.

(Take $\bigvee \{\} = \{\}$, or rewrite the whole thing.)

\emph{Proof.} ($\Rightarrow$) By induction on the $\sstl$ proof tree for $\Gamma \Rightarrow \Delta$, for any shorter $\sstl$ proof of $\Gamma' \Rightarrow \Delta'$ we have a $\stl$ proof of $\Gamma \Rightarrow \bigvee \Delta$ from the induction hypothesis.

\begin{enumerate}
  \item[Axioms] are trivial.
  \item[$R \vee_1$] We have a $\sstl$ proof tree for $\Gamma \Rightarrow A \vee B, \Delta$ and a shorter one for $\Gamma \Rightarrow A, \Delta$. The induction hypothesis with $\Delta' = A, \Delta$ gives us a proof tree for $\Gamma \Rightarrow A \vee \bigvee \Delta$. Apply $R \vee_1$ in $\sstl$ then use $cut$ into the commutativity and associativity of $\vee$ to get $\Gamma \Rightarrow (A \vee B) \vee \bigvee \Delta$.
  \item[$R \vee_2$] Similar to the previous case, and this time we won't need commutativity.
  \item[$R \wedge$] IH twice with $\Delta' = A, \Delta$ and $\Delta' = B, \Delta$. Apply $R \wedge$ in $\stl$ to get $\Gamma \Rightarrow (A \vee \bigvee \Delta) \wedge (B \vee \bigvee \Delta)$, then $cut$ into the distributivity of $\vee$ over $\wedge$ to get $\Gamma \Rightarrow (A \wedge B) \vee \bigvee \Delta$.
  \item[$R \rightarrow$] IH with $\Delta' = B$, then apply $R \rightarrow$ in $\stl$.
  \item[$R w$] IH with $\Delta' = \Delta$, which gives us a proof for $\Gamma \Rightarrow \bigvee \Delta$ in $\stl$, then apply $R \vee$ to get $\Gamma \Rightarrow A \vee \bigvee \Delta$.
  \item[$R c$] IH with $\Delta' = A, A, \Delta$, which gives a proof for $\Gamma \Rightarrow A \vee A \vee \bigvee \Delta$, then $cut$ into $A \vee A \vee \bigvee \Delta \Rightarrow A \vee \bigvee \Delta$.
  \item[$N$] IH with $\Delta' = \Delta$, which gives a prof for $\Gamma \Rightarrow \bigvee \Delta$. Apply $N$ to get $\nabla \Gamma \Rightarrow \nabla \bigvee \Delta$, then $cut$ into the distributivity of $\nabla$ over $\vee$.
  \item[$L \star$] ($\star \in \{ \wedge_1, \wedge_2, \vee, \rightarrow, w, c \}$) IH with $\Delta' = \Delta$ (except for $L \rightarrow$, where we would take $\Delta' = A$ for one of the subtrees), then apply $L \rightarrow$ in $\stl$.
  \item[$cut$] IH twice with $\Delta' = A, \Pi$ and $\Delta' = \Delta$ which gives us two proofs for $\Gamma \Rightarrow A \vee \bigvee \Pi$ and $\Sigma, A \Rightarrow \bigvee \Delta$. Deduce $\Sigma, A \vee \bigvee \Pi \Rightarrow \bigvee \bigvee \Pi \vee \Delta$ from the latter sequent, the $cut$ it with the first one to get $\Gamma, \Sigma \Rightarrow \bigvee \bigvee \Pi \vee \Delta$.
\end{enumerate}

($\Leftarrow$) By induction on the $\stl$ proof tree for $\Gamma \Rightarrow \bigvee \Delta$, for any shorter $\stl$ proof of $\Gamma' \Rightarrow \bigvee \Delta'$ we have a $\sstl$ proof of $\Gamma \Rightarrow \Delta$ from the induction hypothesis. Notice that the in all cases we can only take $\Delta$ to be a singleton ($\bigvee \{ A \} = A$) except in $R \vee_1$ and $R \vee_2$.

\begin{enumerate}
  \item[Axioms] are trivial.
  \item[$R \vee_1$] We have a $\stl$ proof tree for $\Gamma \Rightarrow A \vee B$ and a shorter one for $\Gamma \Rightarrow A$. Take $\Delta' = A$ and use induction hypothesis to get a $\sstl$ proof of $\Gamma \Rightarrow A$. Now we have two subcases: If we take $\Delta = \{ A \vee B \}$, then we can apply $R \vee_1$ in $\sstl$ to get $\Gamma \Rightarrow A \vee B$. If we take $\Delta = \{ A, B \}$, then by $R w$ in $\sstl$ we would get $\Gamma \Rightarrow A, B$.
  \item[$R \vee_2$] Similar to the previous case.
  \item[$R \wedge$] IH twice with $\Delta' = A$ and $\Delta' = B$, then apply $R \wedge$ in $\sstl$.
  \item[$R \rightarrow$] IH with $\Delta' = B$, then apply $R \rightarrow$ in $\sstl$. Notice that this rule is single succedent.
  \item[$R w$] IH with $\Delta' = \{\}$ and then apply $\sstl$'s $R w$.
  \item[$N$] IH with $\Delta' = \delta$ and then apply $\sstl$'s $N$.
  \item[$L \star$] ($\star \in \{ \wedge_1, \wedge_2, \vee, \rightarrow, w, c \}$) IH with $\Delta' = \delta$ (except for $R \rightarrow$ where we use IH once with $\Delta' = A$), then apply $R \star$ in $\sstl$.
  \item[$cut$] IH twice with $\Delta' = A$ and $\Delta' = \delta$, then apply $cut$ in $\sstl$.
\end{enumerate}
\begin{flushright}$\Box$\end{flushright}

\section{Tableau system for STL} A \emph{tableau} for $\stl$ is a tree structure with following cnostructors. Remember that $\Phi$ and $\Psi$ are sets of \emph{signed} formulas, i.e., formulas with a prefix $T$ or $F$. We will use the same comma notation for set union and singleton as we used for multisets.
(Tableaux here are drawn upside down, due to technical problems.)

\begin{multicols}{3}
  \begin{prooftree}
    \AXC{$\bigotimes$}
    \RightLabel{$Cl$}
    \UIC{$\Phi, T A, F A$}
  \end{prooftree}
  \columnbreak
  \begin{prooftree}
    \AXC{$\bigotimes$}
    \RightLabel{$FT$}
    \UIC{$\Phi, F \top$}
  \end{prooftree}
  \columnbreak
  \begin{prooftree}
    \AXC{$\bigotimes$}
    \RightLabel{$TF$}
    \UIC{$\Phi, T \bot$}		
  \end{prooftree}
\end{multicols}

\begin{multicols}{2}
  \begin{prooftree}
    \AXC{$\Phi, F A$}
    \AXC{$\Phi, F B$}
    \RightLabel{$F \wedge$}
    \BIC{$\Phi, F A \wedge B$}
  \end{prooftree}
  \columnbreak
  \begin{prooftree}
    \AXC{$\Phi, T A, T B$}
    \RightLabel{$T \wedge$}
    \UIC{$\Phi, T A \wedge B$}
  \end{prooftree}
\end{multicols}

\begin{multicols}{2}
  \begin{prooftree}
    \AXC{$\Phi, T A$}
    \AXC{$\Phi, T B$}
    \RightLabel{$T \vee$}
    \BIC{$\Phi, T A \vee B$}
  \end{prooftree}
  \columnbreak
  \begin{prooftree}
    \AXC{$\Phi, F A, F B$}
    \RightLabel{$F \vee$}
    \UIC{$\Phi, F A \vee B$}
  \end{prooftree}
\end{multicols}

 \begin{multicols}{2}
  \begin{prooftree}
    \AXC{$\Phi_T, F A$}
    \AXC{$\Phi, T B$}
    \RightLabel{$T \rightarrow$}
    \BIC{$\Phi, T \nabla (A \rightarrow B)$}
  \end{prooftree}
  \columnbreak
  \begin{prooftree}
    \AXC{$\nabla \Phi_T, T A, F B$}
    \RightLabel{$F \rightarrow$}
    \UIC{$\Phi, F A \rightarrow B$}
  \end{prooftree}
\end{multicols}


 \begin{prooftree}
   \AXC{$\Phi$}
   \RightLabel{$Th$}
   \UIC{$\Phi, \Psi$}
 \end{prooftree}
 
 \begin{prooftree}
   \AXC{$\Phi$}
   \RightLabel{$N$}
    \UIC{$\nabla \Phi$}
 \end{prooftree}

\begin{center}
 \begin{prooftree}
   \AXC{$\Phi, T A$}
   \AXC{$\Psi, F A$}
   \RightLabel{$cut$}
   \BIC{$\Phi, \Psi$}
 \end{prooftree}
\end{center}

A branch of a tableau which ends with either $Cl$, $FT$ or $TF$ is a \emph{closed branch}, otherwise it is \emph{open}. A tableau with no open branch is a \emph{closed tableau}.

\section{Theorem} $\stl \vdash \Gamma \Rightarrow \Delta$ \textit{iff there exists a closed tableau for} $T \Gamma, F \Delta$.

\emph{Proof.} ($\Rightarrow$) By induction on the $\stl$ proof of $\Gamma \Rightarrow \Delta$, for any shorter proof of $\Gamma' \Rightarrow \Delta'$ we would have a closed tableau for $T \Gamma, F \Delta$ from the inducion hypothersis.

\begin{enumerate}
  \item[$R \wedge$] We have a $\stl$ proof of $\Gamma \Rightarrow A \wedge B, \Delta$ and a two shorter ones for $\Gamma \Rightarrow A, \Delta$ and $\Gamma \Rightarrow B, \Delta$. The induction hypothesis gives us two closed tableaux for $T \Gamma, F A, F \Delta$ and $T \Gamma, F B, F \Delta$. We can construct a closed tableau for $T \Gamma, F A \wedge B, F \Delta$ using $F \wedge$.
  \item[$R \vee_1$] From IH we have a closed tableau for $T \Gamma, F A, F \Delta$. Use $Th$ to get a closed tableau for $T \Gamma, F A, F B, F \Delta$, then $F \vee$ to get $T \Gamma, F A \vee B, F \Delta$.
  \item[$R \vee_2$] Similar.
  \item[$R \rightarrow$] IH gives us a closed tableau for $T \nabla \Gamma, T A, F B$. Construct a closed tableau for $T \Gamma, F A \rightarrow B$ using $F \rightarrow$ with $\Phi = \Phi_T = T \Gamma$
  \item[$R w$] IH gives us a closed tableau for $T \Gamma, F \Delta$. Construct a closed tableau for $T \Gamma, T A, F \Delta$ using $Th$.
  \item[$R c$] The closed tableau that we would get from IH should be enough, since $T \Gamma, F A, F A, F \Delta = T \Gamma, F A, F \Delta$.
  \item[$L \wedge_1$] IH gives us a closed tableau for $T \Gamma, T A, F \Delta$. Use $Th$ to get a closed tableau for $T \Gamma, T A, T B, F \Delta$, then use $T \wedge$ to get a closed tableau for $T \Gamma, T A \wedge B, F \Delta$.
  \item[$L \wedge_2$] Similar.
  \item[$L \vee$] IH gives us two closed tableaux for $T \Gamma, T A, F \Delta$ and $T \Gamma, T B, F \Delta$. Use $T \vee$ to construct a closed tableau for $T \Gamma, T A \vee B, F \Delta$.
  \item[$L \rightarrow$] IH gives us two closed tableaux for $T \Gamma, F A$ and $T \Gamma, T B, F \Delta$. Use $T \rightarrow$ with $\Phi = T \Gamma, F \Delta$ to construct a closed tableau for $T \Gamma, T \nabla (A \rightarrow B), F \Delta$.
  \item[$L w$] Similar to the case for $R w$.
  \item[$L c$] Similar to the case for $R c$.
  \item[$N$] IH gives us a closed tableau for $T \Gamma, F \Delta$ which turns to closed tableau for $T \nabla \Gamma, F \nabla \Delta$.
  \item[$cut$] IH gives us two closed tableaux for $T \Gamma, F A, F \Pi$ and $T \Sigma, T A, F \Delta$. Use $cut$ to get a closed tableau for $T \Gamma, F \Pi, T \Sigma, F \Delta$.
\end{enumerate}

($\Leftarrow$) Suppose there exists a closed tableau for $T \Gamma, F \Delta$. Proceed by induction on the construction of this tableau.

\begin{enumerate}
  \item[$T \wedge$] There is a closed tableau for $\Phi, T A, T B$ and a shorter one for $\Phi, T A \wedge B$. By induction hypothesis we have a $\sstl$ proof tree for $T_\Phi, A, B \Rightarrow F_\Phi$. Apply $L \wedge_1$, $L \wedge_2$ and then $L c$ to get $T_\Phi, A \wedge B \Rightarrow F_\Phi$.
  \item[$T \vee$] We have two closed tableaux for $\Phi, T A$ and $\Phi, T B$. IH gives us two proof trees for $T_\Phi, A \Rightarrow F_\Phi$ and $T_\Phi, B \Rightarrow F_\Phi$. Apply $L \vee$ to get $T_\Phi, A \vee B \Rightarrow F_\Phi$.
  \item[$T \rightarrow$] We have two closed tableaux for $\Phi_T, F A$ and $\Phi, T B$. IH gives us two proof trees for $T_\Phi \Rightarrow A$ and $T_\Phi, B \Rightarrow F_\Phi$. Apply $L \rightarrow$ to get $T_\Phi, \nabla (A \rightarrow B) \Rightarrow F_\Phi$.
  \item[$F \wedge$] There exists two closed tableaux for $\Phi, F A$ and $\Phi, F B$, for which IH gives two proof trees for $T_\Phi \Rightarrow A, F_\Phi$ and $T_\Phi \Rightarrow B, F_\Phi$. Apply $R \wedge$ to get $T_\Phi \Rightarrow A \wedge B, F_\Phi$.
  \item[$F \vee$] IH gives a proof for $T_\Phi \Rightarrow A, B, F_\Phi$. Apply $R \vee_1$, $R \vee_2$ and then $R c$.
  \item[$F \rightarrow$] There is a proof tree for $\nabla T_\Phi, A \Rightarrow B$ from IH. By $R \rightarrow$ we would have $T_\Phi \Rightarrow A \rightarrow B$. Now, using $R w$, we can add $F_\Phi$ to the right of the sequent.
  \item[$N$] We have a closed tableau for $\nabla \Phi$ and a shorter one for $\Phi$. IH gives us a proof tree for $T_\Phi \Rightarrow F_\Phi$. Apply $N$ in $\sstl$ to get $\nabla T_\Phi \Rightarrow \nabla F_\Phi$.
  \item[$Th$] IH, $L w$ and $R w$.
  \item[$cut$] IH two times, and then $cut$ in $\sstl$.
\end{enumerate}
\begin{flushright}$\Box$\end{flushright}


\section{Cut-free system for STL} \quad \\

 \begin{multicols}{3}
   \begin{prooftree}
     \AXC{}
     \RightLabel{$Id$}
     \UIC{$ A \Rightarrow A$}
   \end{prooftree}
   \columnbreak
   \begin{prooftree}
     \AXC{}
     \RightLabel{$Ta$}
     \UIC{$ \Rightarrow \top$}
   \end{prooftree}
   \columnbreak
   \begin{prooftree}
     \AXC{}
     \RightLabel{$Ex$}
     \UIC{$ \nabla^n \bot \Rightarrow $}		
   \end{prooftree}
 \end{multicols}
 
 \begin{multicols}{3}
   \begin{prooftree}
     \AXC{$ \Gamma, \nabla^n A \Rightarrow \delta$}
     \RightLabel{$L \wedge_1$}
     \UIC{$ \Gamma, \nabla^n (A \wedge B) \Rightarrow \delta$}		
   \end{prooftree}
   \columnbreak
   \begin{prooftree}
     \AXC{$ \Gamma, \nabla^n B \Rightarrow \delta$}
     \RightLabel{$L \wedge_2$}
     \UIC{$\Gamma, \nabla^n (A \wedge B) \Rightarrow \delta$}		
   \end{prooftree}
   \columnbreak
   \begin{prooftree}
     \AXC{$\Gamma \Rightarrow A$}
     \AXC{$\Gamma \Rightarrow B$}
     \RightLabel{$R \wedge$}
     \BIC{$ \Gamma \Rightarrow A \wedge B$}		
   \end{prooftree}
 \end{multicols}
 
 \begin{multicols}{3}
   \begin{prooftree}
     \AXC{$ \Gamma, \nabla^n A \Rightarrow \delta$}
     \AXC{$\Gamma, \nabla^n B \Rightarrow \delta$}
     \RightLabel{$L \vee$}
     \BIC{$ \Gamma, \nabla^n (A \vee B) \Rightarrow \delta$}		
   \end{prooftree}
   \columnbreak
   \begin{prooftree}
     \AXC{$\Gamma \Rightarrow A$}
     \RightLabel{$R \vee_1$}
     \UIC{$\Gamma \Rightarrow A \vee B$}		
   \end{prooftree}
   \columnbreak
   \begin{prooftree}
     \AXC{$\Gamma \Rightarrow B$}
     \RightLabel{$R \vee_2$}
     \UIC{$\Gamma \Rightarrow A \vee B$}		
   \end{prooftree}
 \end{multicols}
 
  \begin{multicols}{2}
   \begin{prooftree}
     \AXC{$\Gamma \Rightarrow \nabla^n A$}
     \AXC{$\Gamma, \nabla^n B \Rightarrow \delta$}
     \RightLabel{$L \rightarrow$}
     \BIC{$\Gamma, \nabla^{n+1} (A \rightarrow B) \Rightarrow \delta$}		
   \end{prooftree}
   \columnbreak
   \begin{prooftree}
     \AXC{$\nabla \Gamma, A \Rightarrow B$}
     \RightLabel{$R \rightarrow$}
     \UIC{$\Gamma \Rightarrow A \rightarrow B$}		
   \end{prooftree}
 \end{multicols}
 
 \begin{multicols}{2}
  \begin{prooftree}
    \AXC{$ \Gamma \Rightarrow \delta$}
    \RightLabel{$L w$}
    \UIC{$ \Gamma, A \Rightarrow \delta$}
  \end{prooftree}
  \columnbreak
  \begin{prooftree}
    \AXC{$ \Gamma \Rightarrow$}
    \RightLabel{$R w$}
     \UIC{$\Gamma \Rightarrow A$}		
  \end{prooftree}
\end{multicols}

\begin{multicols}{1}
  \begin{prooftree}
    \AXC{$ \Gamma, A, A \Rightarrow \delta$}
    \RightLabel{$Lc$}
    \UIC{$\Gamma, A \Rightarrow \delta$}		
  \end{prooftree}
\end{multicols}

 \begin{prooftree}
   \AXC{$\Gamma \Rightarrow \delta$}
   \RightLabel{$N$}
   \UIC{$\nabla \Gamma \Rightarrow \nabla \delta$}
 \end{prooftree}

 \begin{center}
  \begin{prooftree}
    \AXC{$ \Gamma \Rightarrow A$}
    \AXC{$\Sigma, A \Rightarrow \delta$}
    \RightLabel{$cut$}
    \BIC{$\Gamma, \Sigma \Rightarrow \delta$}
  \end{prooftree}
\end{center}

By $\gstl \vdash \Gamma \Rightarrow \delta$ we mean that the sequent $\Gamma \Rightarrow \delta$ provable in this system.
\pagebreak

\section{Symmetric cut-free system for STL} \quad \\

 \begin{multicols}{3}
   \begin{prooftree}
     \AXC{}
     \RightLabel{$Id$}
     \UIC{$A \Rightarrow A$}
   \end{prooftree}
   \columnbreak
   \begin{prooftree}
     \AXC{}
     \RightLabel{$Ta$}
     \UIC{$\Rightarrow \top$}
   \end{prooftree}
   \columnbreak
   \begin{prooftree}
     \AXC{}
     \RightLabel{$Ex$}
     \UIC{$\nabla^n \bot \Rightarrow $}		
   \end{prooftree}
 \end{multicols}
 
 \begin{multicols}{3}
   \begin{prooftree}
     \AXC{$ \Gamma, \nabla^n A \Rightarrow \Delta$}
     \RightLabel{$L \wedge_1$}
     \UIC{$ \Gamma, \nabla^n (A \wedge B) \Rightarrow \Delta$}		
   \end{prooftree}
   \columnbreak
   \begin{prooftree}
     \AXC{$ \Gamma, \nabla^n B \Rightarrow \Delta$}
     \RightLabel{$L \wedge_2$}
     \UIC{$\Gamma, \nabla^n (A \wedge B) \Rightarrow \Delta$}		
   \end{prooftree}
   \columnbreak
   \begin{prooftree}
     \AXC{$\Gamma \Rightarrow A, \Delta$}
     \AXC{$\Gamma \Rightarrow B, \Delta$}
     \RightLabel{$R \wedge$}
     \BIC{$ \Gamma \Rightarrow A \wedge B, \Delta$}		
   \end{prooftree}
 \end{multicols}
 
 \begin{multicols}{3}
   \begin{prooftree}
     \AXC{$ \Gamma, \nabla^n A \Rightarrow \Delta$}
     \AXC{$\Gamma, \nabla^n B \Rightarrow \Delta$}
     \RightLabel{$L \vee$}
     \BIC{$ \Gamma, \nabla^n (A \vee B) \Rightarrow \Delta$}		
   \end{prooftree}
   \columnbreak
   \begin{prooftree}
     \AXC{$\Gamma \Rightarrow A, \Delta$}
     \RightLabel{$R \vee_1$}
     \UIC{$\Gamma \Rightarrow A \vee B, \Delta$}		
   \end{prooftree}
   \columnbreak
   \begin{prooftree}
     \AXC{$\Gamma \Rightarrow B, \Delta$}
     \RightLabel{$R \vee_2$}
     \UIC{$\Gamma \Rightarrow A \vee B, \Delta$}		
   \end{prooftree}
 \end{multicols}
 
  \begin{multicols}{2}
   \begin{prooftree}
     \AXC{$\Gamma \Rightarrow \nabla^n A$}
     \AXC{$\Gamma, \nabla^n B \Rightarrow \Delta$}
     \RightLabel{$L \rightarrow$}
     \BIC{$\Gamma, \nabla^{n+1} (A \rightarrow B) \Rightarrow \Delta$}		
   \end{prooftree}
   \columnbreak
   \begin{prooftree}
     \AXC{$\nabla \Gamma, A \Rightarrow B$}
     \RightLabel{$R \rightarrow$}
     \UIC{$\Gamma \Rightarrow A \rightarrow B$}		
   \end{prooftree}
 \end{multicols}
 
 \begin{multicols}{2}
  \begin{prooftree}
    \AXC{$ \Gamma \Rightarrow \Delta$}
    \RightLabel{$L w$}
    \UIC{$ \Gamma, A \Rightarrow \Delta$}
  \end{prooftree}
  \columnbreak
  \begin{prooftree}
    \AXC{$ \Gamma \Rightarrow \Delta$}
    \RightLabel{$R w$}
     \UIC{$\Gamma \Rightarrow A, \Delta$}		
  \end{prooftree}
\end{multicols}

\begin{multicols}{2}
  \begin{prooftree}
    \AXC{$ \Gamma, A, A \Rightarrow \Delta$}
    \RightLabel{$Lc$}
    \UIC{$\Gamma, A \Rightarrow \Delta$}		
  \end{prooftree}
  \columnbreak
  \begin{prooftree}
    \AXC{$ \Gamma \Rightarrow A, A, \Delta$}
    \RightLabel{$Rc$}
    \UIC{$\Gamma \Rightarrow A, \Delta$}		
  \end{prooftree}
\end{multicols}

 \begin{prooftree}
   \AXC{$\Gamma \Rightarrow \Delta$}
   \RightLabel{$N$}
   \UIC{$\nabla \Gamma \Rightarrow \nabla \Delta$}
 \end{prooftree}

 \begin{center}
  \begin{prooftree}
    \AXC{$ \Gamma \Rightarrow A, \Pi$}
    \AXC{$\Sigma, A \Rightarrow \Delta$}
    \RightLabel{$cut$}
    \BIC{$\Gamma, \Sigma \Rightarrow \Pi, \Delta$}
  \end{prooftree}
\end{center}


By $\gsstl \vdash \Gamma \Rightarrow \Delta$ we mean that the sequent $\Gamma \Rightarrow \Delta$ provable in this system.

\section{Theorem} \textit{Symmetric and asymmetric cut-free systems are equivalent:}

\quad $\sstl \vdash \Gamma \Rightarrow \Delta$ \quad iff \quad $\stl \vdash \Gamma \Rightarrow \bigvee \Delta$.

\emph{Proof.} Exactly like the previous therem.
\begin{flushright}$\Box$\end{flushright}

\section{Tableau system for cut-free STL} A cut-free \emph{tableau} for $\stl$ is a tree structure with following cnostructors.
(Tableaux here are drawn upside down, due to technical problems.)

\begin{multicols}{3}
  \begin{prooftree}
    \AXC{$\bigotimes$}
    \RightLabel{$Cl$}
    \UIC{$\Phi, T A, F A$}
  \end{prooftree}
  \columnbreak
  \begin{prooftree}
    \AXC{$\bigotimes$}
    \RightLabel{$FT$}
    \UIC{$\Phi, F \top$}
  \end{prooftree}
  \columnbreak
  \begin{prooftree}
    \AXC{$\bigotimes$}
    \RightLabel{$TF$}
    \UIC{$\Phi, T \nabla^n \bot$}		
  \end{prooftree}
\end{multicols}

\begin{multicols}{2}
  \begin{prooftree}
    \AXC{$\Phi, F A$}
    \AXC{$\Phi, F B$}
    \RightLabel{$F \wedge$}
    \BIC{$\Phi, F A \wedge B$}
  \end{prooftree}
  \columnbreak
  \begin{prooftree}
    \AXC{$\Phi, T \nabla^n A, T \nabla^n B$}
    \RightLabel{$T \wedge$}
    \UIC{$\Phi, T \nabla^n (A \wedge B)$}
  \end{prooftree}
\end{multicols}

\begin{multicols}{2}
  \begin{prooftree}
    \AXC{$\Phi, T \nabla^n A$}
    \AXC{$\Phi, T \nabla^n B$}
    \RightLabel{$T \vee$}
    \BIC{$\Phi, T \nabla^n (A \vee B)$}
  \end{prooftree}
  \columnbreak
  \begin{prooftree}
    \AXC{$\Phi, F A, F B$}
    \RightLabel{$F \vee$}
    \UIC{$\Phi, F A \vee B$}
  \end{prooftree}
\end{multicols}

 \begin{multicols}{2}
  \begin{prooftree}
    \AXC{$\Phi_T, F \nabla^n A$}
    \AXC{$\Phi, T \nabla^n B$}
    \RightLabel{$T \rightarrow$}
    \BIC{$\Phi, T \nabla^{n+1} (A \rightarrow B)$}
  \end{prooftree}
  \columnbreak
  \begin{prooftree}
    \AXC{$\nabla \Phi_T, T A, F B$}
    \RightLabel{$F \rightarrow$}
    \UIC{$\Phi, F A \rightarrow B$}
  \end{prooftree}
\end{multicols}


 \begin{prooftree}
   \AXC{$\Phi$}
   \RightLabel{$Th$}
   \UIC{$\Phi, \Psi$}
 \end{prooftree}
 
 \begin{prooftree}
   \AXC{$\Phi$}
   \RightLabel{$N$}
    \UIC{$\nabla \Phi$}
 \end{prooftree}


A branch of a tableau which ends with either $Cl$, $FT$ or $TF$ is a \emph{closed branch}, otherwise it is \emph{open}. A tableau with no open branch is a \emph{closed tableau}.

\section{Theorem} $\stl \vdash \Gamma \Rightarrow \Delta$ \textit{iff there exists a closed tableau for} $T \Gamma, F \Delta$.

\emph{Proof.} Same as above.
\begin{flushright}$\Box$\end{flushright}

\end{document}

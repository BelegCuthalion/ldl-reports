\documentclass[a4paper, 12pt]{paper}
\usepackage{amsmath}
\usepackage{amssymb}
\usepackage{mathabx}
\usepackage{titlesec}
\usepackage{fullpage}
\usepackage{tikz-cd}
\usepackage{rotating}
\usepackage{pdflscape}
\usepackage{multicol}
\usepackage{multirow}
\usepackage{diagbox}
\usepackage[left=.5in,right=.5in,top=.5in,bottom=.5in]{geometry}
\usepackage{enumitem}
\usepackage[colorlinks, hypertexnames=false]{hyperref}
\usepackage{bussproofs}
\usepackage[normalem]{ulem}

\setitemize{topsep=3pt,parsep=5pt,partopsep=0pt,label=,leftmargin=1.3pc}
\titleformat{\section}[runin]{\normalfont\bfseries}{\thesection}{0.5em}{}
\titlespacing{\section}{0pc}{5ex plus .1ex minus .2ex}{1pc}
\titleformat{\subsection}[runin]{\normalfont\bfseries}{\thesubsection}{0.7em}{}
\titlespacing{\subsection}{0pc}{2ex plus .1ex minus .2ex}{1pc}
\titleformat{\subsubsection}[runin]{\normalfont\bfseries}{\thesubsubsection}{0.7em}{}
\titlespacing{\subsubsection}{0pc}{2ex plus .1ex minus .2ex}{1pc}
\newcommand\eqn{\refstepcounter{equation}\tag{\theequation}}
\binoppenalty=\maxdimen
\relpenalty=\maxdimen
\newcommand{\ul}{\ulcorner}
\newcommand{\ur}{\urcorner}
\newcommand{\val}[1]{\ulcorner len1 \urcorner}
\newcommand{\caseref}[1]{\hyperref[#1]{\ref{#1}}}
\newcommand{\rot}{\rotatebox{90}}
\newcommand{\p}{\partial}
\newcommand{\todo}[1]{{\color{red}\textbf{TODO} #1}}
\newcommand{\red}{\color{red}}
\newcommand{\stl}{\mathbf{STL}}
\newcommand{\st}{\text{ST}}
\newcommand{\qst}{\mathbf{QST}}
\newcommand{\gstl}{\mathbf{GSTL}}
\newcommand{\gst}{\text{GST}}
\newcommand{\istl}{\mathbf{iSTL}}
\newcommand{\ist}{\text{iST}}
\newcommand{\igstl}{\mathbf{iGSTL}}
\newcommand{\igst}{\text{iGST}}
\newcommand{\sstl}{\mathbb{S}\mathbf{TL}}
\newcommand{\gsstl}{\mathbf{G}\mathbb{S}\mathbf{TL}}
\newcommand{\stt}{\mathbf{ST3}}
\newcommand{\qstt}{\mathbf{QST3}}
\newcommand{\gstt}{\text{GST3}}
\newcommand{\gsttl}{\mathbf{GST3L}}
\newcommand{\istt}{\text{iST3}}
\newcommand{\igstt}{\text{iGST3}}
\newcommand{\D}{\mathcal{D}}
\newcommand{\IH}{\text{IH}}
\newcommand{\qed}{{\begin{flushright}$\Box$\end{flushright}}}
\newtheorem{proposition}{}[section]
\EnableBpAbbreviations

\begin{document}
{\noindent
	v 1.3 \\
{\large\textbf{Definition for $\qst$ and a proof sketch for its cut-elimination theorem and existence of interpolants}}
}
\\

\section{Definitions}

\subsection{$\qst$} is the extension of $\st$ with these rules for first-order quantifiers.

\begin{multicols}{2}
  \begin{prooftree}
    \AXC{$\Gamma, A[t/x] \Rightarrow \Delta$}
    \RightLabel{$L \forall$}
    \UIC{$\Gamma, \nabla \forall x A \Rightarrow \Delta$}
  \end{prooftree}
  \columnbreak
  \begin{prooftree}
    \AXC{$\nabla \Gamma \Rightarrow  A[y/x]$}
    \RightLabel{$R \forall$}
    \UIC{$\Gamma \Rightarrow \forall x A$}
  \end{prooftree}
  \center
  \emph{Provided that $y$ is not free in the conclusion.}
\end{multicols}\vspace*{1em}
\begin{multicols}{2}
  \begin{prooftree}
    \AXC{$\Gamma, A[y/x] \Rightarrow \Delta$}
    \RightLabel{$L \exists$}
    \UIC{$\Gamma, \exists x A \Rightarrow \Delta$}
  \end{prooftree}
  \center
  \emph{Provided that $y$ is not free in the conclusion.}
  \columnbreak
  \begin{prooftree}
    \AXC{$\Gamma \Rightarrow  A[t/x]$}
    \RightLabel{$R \exists$}
    \UIC{$\Gamma \Rightarrow \exists x A$}
  \end{prooftree}
\end{multicols}

\subsection{$\qldl$} is the extension of $\ldl$ with the following rules.

\begin{multicols}{2}
  \begin{prooftree}
    \AXC{$\Gamma, \nabla^{n+1} \forall x A, \nabla^n A[t/x] \Rightarrow \Delta$}
    \RightLabel{$L \forall$}
    \UIC{$\Gamma, \nabla^{n+1} \forall x A \Rightarrow \Delta$}
  \end{prooftree}
  \columnbreak
  \begin{prooftree}
    \AXC{$\nabla \Gamma \Rightarrow  A[y/x]$}
    \RightLabel{$R \forall$}
    \UIC{$\Gamma \Rightarrow \forall x A$}
  \end{prooftree}
  \center
  \emph{Provided that $y$ is not free in the conclusion.}
\end{multicols}\vspace*{1em}
\begin{multicols}{2}
  \begin{prooftree}
    \AXC{$\Gamma, \nabla^n A[y/x] \Rightarrow \Delta$}
    \RightLabel{$L \exists$}
    \UIC{$\Gamma, \nabla^n \exists x A \Rightarrow \Delta$}
  \end{prooftree}
  \center
  \emph{Provided that $y$ is not free in the conclusion.}
  \columnbreak
  \begin{prooftree}
    \AXC{$\Gamma \Rightarrow  A[t/x]$}
    \RightLabel{$R \exists$}
    \UIC{$\Gamma \Rightarrow \exists x A$}
  \end{prooftree}
\end{multicols}

$\qldl_0$ is the system $\qldl$ with all $n = 0$.

$S^-$ is the system $S$ without $cut$.

\section{$\qst \dashv \vdash \qldl$}

\subsection{Theorem} $\qst^- \dashv \vdash \qldl_0^-$.\\
\emph{Proof.} All rules in either systems are admissible in eachother, courtesy of the following lemmas.

\subsubsection{Lemma} $(Id)$ $\qldl_0^- \vdash \Gamma, A \Rightarrow A$.\\
\emph{Proof.} It remains to check two cases.

($\forall x B$)
\begin{prooftree}
  \AXC{} \RightLabel{IH}
  \UIC{$\Gamma, B[y / x] \Rightarrow B[y / x]$}
  \RightLabel{$L \forall$}
  \UIC{$\Gamma, \forall x B \Rightarrow B[y / x]$}
  \RightLabel{$R \forall$}
  \UIC{$\Gamma, \forall x B \Rightarrow \forall x B$}
\end{prooftree}


($\exists x B$)
\begin{prooftree}
  \AXC{} \RightLabel{IH}
  \UIC{$\Gamma, B[y / x] \Rightarrow B[y / x]$}
  \RightLabel{$R \exists$}
  \UIC{$\Gamma, B[y / x] \Rightarrow \exists x B$}
  \RightLabel{$L \exists$}
  \UIC{$\Gamma, \exists x B \Rightarrow \exists x B$}
\end{prooftree}

\subsubsection{Lemma} \emph{(Inversion)} $\qldl_0^- \vdash \Gamma, \exists x A \Rightarrow \Delta$, then $\qldl_0^- \vdash \Gamma, A[y / x] \Rightarrow \Delta$ by a proof-tree of the same height, for all $y$ not occurring freely in $\Gamma$, $\Delta$ or $A$.\\
\emph{Proof.} By induction on the proof-tree $\D$ for $\Gamma, \exists x A \Rightarrow \Delta$. The cases where $\exists x A$ is not principal are handled by commuting the last rule with IH. The case where the last rule is $L \exists$ and $\exists x A$ is principal is trivial.

\subsubsection{Lemma} \emph{(Lc)} $\qldl_0^- \vdash \Gamma, A^2 \Rightarrow \Delta$, then $\qldl_0^- \vdash \Gamma, A \Rightarrow \Delta$ by a proof tree of the same height.\\
\emph{Proof.} It remains to check two cases.

($L \forall$) Let $A = \forall x A$.
\begin{prooftree}
  \AXC{$\Gamma, (\nabla \forall x A)^2, A[t / x] \Rightarrow \Delta$}
  \RightLabel{$L \forall$}
  \UIC{$\Gamma, (\nabla \forall x A)^2 \Rightarrow \Delta$}
\end{prooftree}
By IH we havve $\Gamma, \nabla \forall x A, A[t / x] \Rightarrow \Delta$. By $L \rightarrow$ we have the desired sequent.\\

($L \exists$) Let $A = \exists x A$.
\begin{prooftree}
  \AXC{$\Gamma, \exists x A, A[y / x] \Rightarrow \Delta$}
  \RightLabel{$L \exists$}
  \UIC{$\Gamma, (\exists x A)^2 \Rightarrow \Delta$}
\end{prooftree}
By Inversion Lemma we have $\Gamma, (A[y / x])^2 \Rightarrow \Delta$. By IH and $L \exists$ we have the desired sequent.

\subsection{Theroem} $\qst \dashv \vdash \qldl_0$.
\emph{Proof.} By induction on the height of the top-most $cut$ instance in either of the proof-trees. Replace the sub-trees of the top-most $cut$ using the fact that $\qst^- \dashv \vdash \qldl_0^-$.

\subsection{Theorem} $\qldl \dashv \vdash \qldl_0$.\\
\emph{Proof.} All rules in $\qldl_0$ are instances of rules in $\qldl$. On the other hand, all rules in $\qldl$ can be shown to be admissible in $\qldl_0$ using $cut$ and the results of the following lemmas.

\subsubsection{Lemma} $\qldl \vdash \nabla^n \forall x A \Rightarrow \forall x \nabla^n A$. \\
\emph{Proof.}\quad
\begin{prooftree}
  \AXC{} \RightLabel{$(Id)$}
  \UIC{$\nabla \forall x A, A[y/x] \Rightarrow A[y/x]$}
  \RightLabel{$L \forall$}
  \UIC{$\nabla \forall x A \Rightarrow A[y/x]$}
  \RightLabel{$N^n$} \doubleLine
  \UIC{$\nabla^{n+1} \forall x A \Rightarrow \nabla^n A[y/x]$}
  \RightLabel{$R \forall$}
  \UIC{$\nabla^n \forall x A \Rightarrow \forall x \nabla^n A$}
\end{prooftree}

\subsubsection{Lemma} $\qldl \vdash \nabla^n \exists x A \Rightarrow \exists x \nabla^n A$. \\
\emph{Proof.} Call the following proof-tree $\D_1$.
\begin{prooftree}
  \AXC{} \RightLabel{$(Id)$}
  \UIC{$\nabla \exists x A \Rightarrow \nabla A[y / x]$}
  \RightLabel{$R \exists$}
  \UIC{$\nabla \exists x A \Rightarrow \exists x \nabla A$}
  \RightLabel{$R \rightarrow, Lw$}
  \UIC{$A[y / x] \Rightarrow \Box \exists x \nabla A$}
  \RightLabel{$L \exists$}
  \UIC{$\exists x A \Rightarrow \Box \exists x \nabla A$}
  \RightLabel{$N$}
  \UIC{$\nabla \exists x A \Rightarrow \nabla \Box \exists x \nabla A$}

  
  \AXC{} \RightLabel{$\nabla \Box$}
  \UIC{$\nabla \Box \exists x \nabla A \Rightarrow \exists x \nabla A$}

  \RightLabel{$cut$}
  \LeftLabel{$\D_1$}
  \BIC{$\nabla \exists x A \Rightarrow \exists x \nabla A$}
\end{prooftree}

Construct $\D_{n+1}$ inductively.

\begin{prooftree}
  \AXC{} \RightLabel{$\D_n$}
  \UIC{$\nabla^n \exists x A \Rightarrow \exists x \nabla^n A$}
  \RightLabel{$N$}
  \UIC{$\nabla^{n+1} \exists x A \Rightarrow \nabla \exists x \nabla^n A$}

  \AXC{} \RightLabel{$\D_1$}
  \UIC{$\nabla \exists x \nabla^n A \Rightarrow \exists x \nabla (\nabla^n A)$}

  \RightLabel{$cut$} \LeftLabel{$\D_{n+1}$}
  \BIC{$\nabla^{n+1} \exists x A \Rightarrow \exists x \nabla^{n+1} A$}
\end{prooftree}

\section{Cut-elimination}

\subsection{Rank} Rank of a formula $A$ is defined as
\[ \rho(A) = \begin{cases}
1 & \quad ; A \in \{ P \cup  \bot, \top \} \\
\rho(B) & \quad ; A = \nabla B \\
max(\rho(B), \rho(C)) + 1 & \quad ; A = B \circ C \quad (\circ \in \{ \land, \lor, \rightarrow \}) \\
\rho(B) + 1 & \quad ; A = QxB \quad (Q \in \{ \forall, \exists \})
\end{cases} \]
Notice that $\nabla$ does not increase the rank of a formula.

We also define rank for rule instances and proof-trees as follows. Rank of an instance of the $cut$ rule with cut-formual $A$ is defined to be the rank of $A$. Rank of any other rule instance is $0$.
For a proof tree $\D$, $\rho(\D)$ is the maximum rank of all of its rule instances.

\subsection{Remark} If $y$ is not free in $t$, then $A[s/y][t/x]$ is equal (syntactically) to $A[t/x][s[t/x]/y]$.

\subsection{Lemma} The following substitution rule is admissible in $\qldl$, for any variable $x$ and any term $t$ free for $x$ in $\Gamma$ and $\Delta$.
\begin{prooftree}
  \AXC{$\Gamma \Rightarrow \Delta$}
  \RightLabel{$[x/t]$}
  \UIC{$\Gamma[t/x] \Rightarrow \Delta[t/x]$}
\end{prooftree}

\emph{Proof.} Induction on the structure of $\Gamma \Rightarrow \Delta$:
The only non-trivial cases are those of the rules for quantifiers. The formulas are defined up to $\alpha$-equivalence, so we can choose the variable bound by the quantifiers such that it is different from $x$ and not free in $t$.

Let $\D$ prove $\Gamma \Rightarrow \exists y A$ and ends with $R \exists$. IH gives a proof of $\Gamma[t/x] \Rightarrow A[s/y][t/x]$, which is equal to $\Gamma[t/x] \Rightarrow A[t/x][s[t/x]/y]$, by the previous Remark and the assumption that $y$ is different from $x$ and not free in $t$. Then, by $R \exists$ we would have $\Gamma[t/x] \Rightarrow \exists y (A[t/x])$, and since $y$ is not free in $t$, $\Gamma[t/x] \Rightarrow \exists y A [t/x]$.

Let $\D$ prove $\Gamma, \nabla^n \exists y A \Rightarrow \Delta$ and ends with $L \exists$. IH gives a proof of $\Gamma[t/x], \nabla^n A[z/y][t/x] \Rightarrow \Delta[t/x]$, (where $z$ does not occure in any of the formulas) which is equal to $\Gamma[t/x], \nabla^n A[t/x][z/y] \Rightarrow \Delta[t/x]$. Then, by $L \exists$ we would have $\Gamma[t/x], \nabla^n \exists y (A[t/x]) \Rightarrow \Delta[t/x]$, and since $y$ is not free in $t$, then we have $\Gamma[t/x], \nabla^n \exists y A [t/x] \Rightarrow \Delta[t/x]$.

Let $\D$ prove $\Gamma \Rightarrow \forall y A$ and ends with $R \forall$. IH gives a proof of $\Gamma[t/x] \Rightarrow A[z/y][t/x]$ (where $z$ does not occure in any of the formulas), which is equal to $\Gamma[t/x] \Rightarrow A[t/x][z/y]$. Then, by $R \forall$ we would have $\Gamma[t/x] \Rightarrow \forall y (A[t/x])$, and since $y$ is not free in $t$, $\Gamma[t/x] \Rightarrow \forall y A [t/x]$.

Let $\D$ prove $\Gamma, \nabla^{n+1} \forall y A \Rightarrow \Delta$ and ends with $L \forall$. IH gives a proof of $\Gamma[t/x], \nabla^n A[s/y][t/x] \Rightarrow \Delta[t/x]$, which is equal to $\Gamma[t/x], \nabla^n A[t/x][s[t/x]/y] \Rightarrow \Delta[t/x]$. Then, by $L \forall$ we would have $\Gamma[t/x], \nabla^{n+1} \forall y (A[t/x]) \Rightarrow$ $\Delta[t/x]$, and since $y$ is not free in $t$, $\Gamma[t/x], \nabla^n \forall y A [t/x] \Rightarrow \Delta[t/x]$.



\subsection{Theorem} If $\qldl$ proves $\Gamma \Rightarrow A$ and $\Sigma , \nabla^n A \Rightarrow \Delta$ with proof-trees of ranks less than $\rho(A)$, then it also proves $\nabla^n \Gamma, \Sigma \Rightarrow \Delta$ also with a proof tree of a rank less than $\rho(A)$.

\emph{Proof.} We have two low-rank proof-trees, $\D_0$ and $\D_1$.
\[
  \genfrac{}{}{0pt}{}{\D_0}{\Gamma \Rightarrow A}
  \hspace{3em}
  \genfrac{}{}{0pt}{}{\D_1}{\Sigma , \nabla^n A \Rightarrow \Delta}
\]
We will construct a low-rank proof-tree, $\D$.
\[\genfrac{}{}{0pt}{}{\D}{\nabla^n \Gamma, \Sigma \Rightarrow \Delta} \]

Use mutual induction on the construction of $\qldl$ proofs of $\Gamma \Rightarrow A$ and $\Sigma , \nabla^n A \Rightarrow \Delta$. All propositional cases are handled as was shown before.

$\bullet$ Let $\D_0$ end with $L \forall$.

\begin{prooftree}
  \AXC{$\D_0'$} \noLine
  \UIC{$\Gamma, \nabla^{r+1} \forall x B, \nabla^r B[t/x] \Rightarrow A$}
  \RightLabel{$L \forall$}
  \UIC{$\Gamma, \nabla^{r+1} \forall x B \Rightarrow A$}

  \AXC{$\D_1$} \noLine
  \UIC{$\Sigma , \nabla^n A \Rightarrow \Delta$}
  \noLine
  \BIC{}
\end{prooftree}

Construct $\D$ as follows.
\begin{prooftree}
  \AXC{$\D_0'$} \noLine
  \UIC{$\Gamma, \nabla^{r+1} \forall x B, \nabla^r B[t/x] \Rightarrow A$}
  \AXC{$\D_1$} \noLine
  \UIC{$\Sigma , \nabla^n A \Rightarrow \Delta$}
  \RightLabel{$IH(\D_0', \D_1)$}
  \BIC{$\nabla^n \Gamma, \nabla^{n + r + 1} \forall x B, \nabla^{n + r} B[t/x], \Sigma \Rightarrow \Delta$}
  \RightLabel{$L \forall$} \doubleLine
  \UIC{$\nabla^n \Gamma , \nabla^{n + r + 1} B, \Sigma \Rightarrow \Delta$}
\end{prooftree}


$\bullet$ Let $\D_0$ end with $L \exists$. (So $y$ does not occur free in its conclusion.)

\begin{prooftree}
  \AXC{$\D_0'$} \noLine
  \UIC{$\Gamma, \nabla^r B[y/x] \Rightarrow A$}
  \RightLabel{$L \exists$}
  \UIC{$\Gamma, \nabla^r \exists x B \Rightarrow A$}

  \AXC{$\D_1$} \noLine
  \UIC{$\Sigma , \nabla^n A \Rightarrow \Delta$}
  \noLine
  \BIC{}
\end{prooftree}

Construct $\D$ as follows. Notice that $IH(\D_0', \D_1)$ does not introduce $y$.
\begin{prooftree}
  \AXC{$\D_0'$} \noLine
  \UIC{$\Gamma, \nabla^r B[y/x] \Rightarrow A$}
  \AXC{$\D_1$} \noLine
  \UIC{$\Sigma , \nabla^n A \Rightarrow \Delta$}
  \RightLabel{$IH(\D_0', \D_1)$}
  \BIC{$\nabla^n \Gamma , \nabla^{n + r} B[t/x] , \Sigma \Rightarrow \Delta$}
  \RightLabel{$L \exists$} \doubleLine
  \UIC{$\nabla^n \Gamma , \nabla^{n + r} B , \Sigma \Rightarrow \Delta$}
\end{prooftree}


$\bullet$ Let $\D_1$ end with $R \forall$. (So $y$ does not occur free in its conclusion.)

\begin{prooftree}
  \AXC{$\D_0$} \noLine
  \UIC{$\Gamma \Rightarrow A$}

  \AXC{$\D_1'$} \noLine
  \UIC{$\nabla \Sigma , \nabla^{n+1} A \Rightarrow B[y/x]$}
  \RightLabel{$R \forall$}
  \UIC{$\Sigma , \nabla^n A \Rightarrow \forall x B$}

  \noLine
  \BIC{}
\end{prooftree}

Construct $\D$ as follows. Notice that $IH(\D_0, \D_1')$ does not introduce $y$.
\begin{prooftree}
  \AXC{$\D_0$} \noLine
  \UIC{$\Gamma \Rightarrow A$}
  \AXC{$\D_1'$} \noLine
  \UIC{$\nabla \Sigma , \nabla^{n+1} A \Rightarrow B[y/x]$}
  \RightLabel{$IH(\D_0, \D_1')$}
  \BIC{$\nabla^{n+1} \Gamma , \nabla \Sigma \Rightarrow B[y/x]$}
  \RightLabel{$R \forall$}
  \UIC{$\nabla^n \Gamma , \Sigma \Rightarrow \forall x B$}
\end{prooftree}


$\bullet$ Let $\D_1$ end with $R \exists$.

\begin{prooftree}
  \AXC{$\D_0$} \noLine
  \UIC{$\Gamma \Rightarrow A$}

  \AXC{$\D_1'$} \noLine
  \UIC{$\Sigma, \nabla^n A \Rightarrow B[t/x]$}
  \RightLabel{$R \exists$}
  \UIC{$\Sigma, \nabla^n A \Rightarrow \exists x B$}

  \noLine
  \BIC{}
\end{prooftree}

Construct $\D$ as follows.

\begin{prooftree}
  \AXC{$\D_0$} \noLine
  \UIC{$\Gamma \Rightarrow A$}
  \AXC{$\D_1'$} \noLine
  \UIC{$\Sigma , \nabla^n A \Rightarrow B[t/x]$}
  \RightLabel{$IH(\D_0, \D_1')$}
  \BIC{$\nabla^n \Gamma , \Sigma \Rightarrow B[t/x]$}
  \RightLabel{$R \exists$}
  \UIC{$\nabla^n \Gamma , \Sigma \Rightarrow \exists x B$}
\end{prooftree}


$\bullet$ Let $\D_1$ end with $L \forall$, but the principal formula is not $\nabla^n A$.

\begin{prooftree}
  \AXC{$\D_0$} \noLine
  \UIC{$\Gamma \Rightarrow A$}

  \AXC{$\D_1'$} \noLine
  \UIC{$\Sigma, \nabla^n A, \nabla^{r+1} \forall x B, \nabla^r B[t/x] \Rightarrow \Delta$}
  \RightLabel{$L \forall$}
  \UIC{$\Sigma, \nabla^n A, \nabla^{r+1} \forall x B \Rightarrow \Delta$}

  \noLine
  \BIC{}
\end{prooftree}

Construct $\D$ as follows.
\begin{prooftree}
  \AXC{$\D_0$} \noLine
  \UIC{$\Gamma \Rightarrow A$}
  \AXC{$\D_1'$} \noLine
  \UIC{$\Sigma, \nabla^n A, \nabla^{r+1} \forall x B, \nabla^r B[t/x] \Rightarrow \Delta$}
  \RightLabel{$IH(\D_0, \D_1')$}
  \BIC{$\nabla^n \Gamma, \Sigma, \nabla^{r+1} \forall x B, \nabla^r B[t/x] \Rightarrow \Delta$}
  \RightLabel{$L \forall$}
  \UIC{$\nabla^n \Gamma, \Sigma, \nabla^{r+1} \forall x B \Rightarrow \Delta$}
\end{prooftree}


$\bullet$ Let $\D_1$ end with $L \exists$, but the principal formula is not $\nabla^n A$.

\begin{prooftree}
  \AXC{$\D_0$} \noLine
  \UIC{$\Gamma \Rightarrow A$}

  \AXC{$\D_1'$} \noLine
  \UIC{$\Sigma, \nabla^n A, \nabla^r B[t/x] \Rightarrow \Delta$}
  \RightLabel{$L \exists$}
  \UIC{$\Sigma, \nabla^n A, \nabla^r \exists x B \Rightarrow \Delta$}

  \noLine
  \BIC{}
\end{prooftree}

Construct $\D$ as follows.
\begin{prooftree}
  \AXC{$\D_0$} \noLine
  \UIC{$\Gamma \Rightarrow A$}
  \AXC{$\D_1'$} \noLine
  \UIC{$\Sigma, \nabla^n A, \nabla^r B[t/x] \Rightarrow \Delta$}
  \RightLabel{$IH(\D_0, \D_1')$}
  \BIC{$\nabla^n \Gamma, \Sigma, \nabla^r B[t/x] \Rightarrow \Delta$}
  \RightLabel{$L \exists$}
  \UIC{$\nabla^n \Gamma, \Sigma, \nabla^r \exists x B \Rightarrow \Delta$}
\end{prooftree}


$\bullet$ Let $\D_0$ end with $R \forall$ and none of the other cases hold; The cut-formula would be of the form $\forall x A$. This implies that $\nabla^n A$ should also be principal in the last rule of $\D_1$, and this rule must be either $N$---which is handeld in propositional cases---or $L \forall$.

\begin{prooftree}
  \AXC{$\D_0'$} \noLine
  \UIC{$\nabla \Gamma \Rightarrow A[y/x]$}
  \RightLabel{$R \forall$}
  \UIC{$\Gamma \Rightarrow \forall x A$}

  \AXC{$\D_1'$} \noLine
  \UIC{$\Sigma, \nabla^{n+1} \forall x A, \nabla^n A[t/x] \Rightarrow \Delta$}
  \RightLabel{$L \forall$}
  \UIC{$\Sigma, \nabla^{n+1} \forall x A \Rightarrow \Delta$}

  \noLine
  \BIC{}
\end{prooftree}

Construct $\D$ as follows. Notice that $y$ does not occur free in the conclusion of $\D_0$, so $\D_0'[t/y]$ proves $\nabla \Gamma [t/y] \Rightarrow A[y/x][t/y]$, which is equal to $\nabla \Gamma \Rightarrow A[t/x]$.

\begin{prooftree}
  \AXC{$\D_0'$} \noLine
  \UIC{$\nabla \Gamma \Rightarrow A$} \RightLabel{$[t/y]$}
  \UIC{$\nabla \Gamma \Rightarrow A[t/x]$}

  \AXC{$\D_0$} \noLine
  \UIC{$\Gamma \Rightarrow \forall x A$}
  
  \AXC{$\D_1'$} \noLine
  \UIC{$\Sigma, \nabla^{n+1} \forall x A, \nabla^n A[t/x] \Rightarrow \Delta$}

  \RightLabel{$IH(\D_0, \D_1')$}
  \BIC{$\nabla^{n+1} \Gamma, \Sigma, \nabla^n A[t/x] \Rightarrow \Delta$}

  \BIC{$(\nabla^{n+1} \Gamma)^2, \Sigma \Rightarrow \Delta$}

  \RightLabel{$(Lc)$} \doubleLine
  \UIC{$\nabla^{n+1} \Gamma, \Sigma \Rightarrow \Delta$}
\end{prooftree}


$\bullet$ Let $\D_0$ end with $R \exists$ and none of the other cases hold; The cut-formula would be of the form $\exists x A$. This implies that a member of the cut-burden should also be principal in the last rule of $\D_1$, and this rule must be either $N$---which is handeld in propositional cases---or $L \exists$.

\begin{prooftree}
  \AXC{$\D_0'$} \noLine
  \UIC{$\Gamma \Rightarrow A[t/x]$}
  \RightLabel{$R \exists$}
  \UIC{$\Gamma \Rightarrow \exists x A$}

  \AXC{$\D_1'$} \noLine
  \UIC{$\Sigma, \nabla^n A[y/x] \Rightarrow \Delta$}
  \RightLabel{$L \exists$}
  \UIC{$\Sigma, \nabla^n \exists x A \Rightarrow \Delta$}

  \noLine
  \BIC{}
\end{prooftree}

Construct $\D$ as follows. Notice that $y$ does not occur free in the conclusion of $\D_1$, so $\D_1'[t/y]$ proves $\Sigma[t/y], \nabla^n A[y/x][t/y] \Rightarrow \Delta[t/y]$, which is equal to $\Sigma, \nabla^n A[t/x] \Rightarrow \Delta$.

\begin{prooftree}
  \AXC{$\D_0'$} \noLine
  \UIC{$\Gamma \Rightarrow A[t/x]$}

  
  \AXC{$\D_1'$} \noLine
  \UIC{$\Sigma, \nabla^n A[y/x] \Rightarrow \Delta$} \RightLabel{$[t/y]$}
  \UIC{$\Sigma, \nabla^n A[t/x] \Rightarrow \Delta$}

  \BIC{$ \nabla^n \Gamma , \Sigma \Rightarrow \Delta$}
\end{prooftree}

\section{Interpolation} $\Gamma_1, \Gamma_2 \Rightarrow \Delta$ then there exists a formula $C$ such that
\begin{enumerate}
  \item $\Gamma_1 \Rightarrow C$
  \item $\Gamma_2, C \Rightarrow \Delta$
  \item $V(C) \subseteq V(\Gamma_1) \cap V(\Gamma_2, \Delta)$.
\end{enumerate}
\emph{Proof.} By induction on the $\qldl$ proof-tree for $\Gamma_1, \Gamma_2 \Rightarrow \Delta$. The cases for propositional rules are handled as before. It remains four cases for $L \forall$, $R \forall$, $L \exists$ and $R \exists$.

\begin{enumerate}
  \item ($L \forall$) \quad \begin{enumerate}
      \item Suppose we want to prove $\Gamma_1, \nabla^{n+1} \forall x A \Rightarrow C$ and $\Gamma_2, C \Rightarrow \Delta$. By induction hypothesis we have $\Gamma_2, C \Rightarrow \Delta$. By induction hypothesis we have $\Gamma_1, \nabla^n A[t / x], \nabla^{n+1} \forall x A \Rightarrow C$. By $L \forall$ we have $\Gamma_1, \nabla^{n+1} \forall x A \Rightarrow C$.
      
      \item Suppose we want to prove $\Gamma_1 \Rightarrow C$ and $\Gamma_2, \nabla^{n+1} \forall x A, C \Rightarrow \Delta$ for some $C$. By induction hypothesis we have $\Gamma_1 \Rightarrow C_2$. By induction hypothesis we have $\Gamma_2, \nabla^n A[t / x], \nabla^{n+1} \forall x A, C \Rightarrow \Delta$, by $L \forall$ we have $\Gamma_2, \nabla^{n+1} \forall x A, C \Rightarrow \Delta$.
    \end{enumerate}
    
    \item ($R \forall$) From induction hypothesis we have $\nabla \Gamma_1 \Rightarrow C$. Take $\top \rightarrow C$ as the desired formula. So we want to prove $\Gamma_1 \Rightarrow \top \rightarrow C$ and $\Gamma_2, \top \rightarrow C \Rightarrow \forall x A$. Using $Lw$ and $R \rightarrow$ we have $\Gamma_1 \Rightarrow \top \rightarrow C$/. From induction hypothesis we also have $\nabla \Gamma_2, C \Rightarrow A[y / x]$, for some $y \not \in \Gamma_1, \Gamma_2, A$. Using $L \rightarrow$ we would have $\nabla \Gamma_2, \nabla (\top \rightarrow C) \Rightarrow A[t / x]$. This won't introduce $y$, so we can use $R \forall$ to get $\Gamma_2, \top \rightarrow C \Rightarrow \forall x A$.
    
    \item ($L \exists$) \begin{enumerate}
      \item Suppose we want to prove $\Gamma_1, \nabla^n \exists x A \Rightarrow C$ and $\Gamma_2, C \Rightarrow \Delta$. From induction hypothesis we have $\Gamma_1, \nabla^n A[y / x] \Rightarrow C$, for some $y \not \in \Gamma_1, \Gamma_2, A$. Apply $L \exists$ to get $\Gamma_1, \nabla^n \exists x A \Rightarrow C$. From induction hypothesis we also have $\Gamma_2, C \Rightarrow \Delta$.
      
      \item Suppose we want to prove $\Gamma_1 \Rightarrow C$ and $\Gamma_2, \nabla^n \exists x A \Rightarrow \Delta$. From induction hypothesis we have $\Gamma_1 \Rightarrow C$ and $\Gamma_2, \nabla^n A[y / x] \Rightarrow \Delta$, for some $y \not \in \Gamma_1, \Gamma_2, A$. Apply $L \exists$ to get $\Gamma_2, \nabla^n \exists x A \Rightarrow \Delta$.
    \end{enumerate}

    \item ($R \exists$) We want to prove $\Gamma_1 \Rightarrow C$ and $\Gamma_2, C \Rightarrow \exists x A$. From induction hypothesis we have $\Gamma_1 \Rightarrow C$ and $\Gamma_2, C \Rightarrow A[t / x]$, on which we can apply $R \exists$ to get $\Gamma_2, C \Rightarrow \exists x A$.
\end{enumerate}


\end{document} 
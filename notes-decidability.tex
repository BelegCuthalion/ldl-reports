\documentclass[a4paper, 12pt]{paper}
\usepackage{amsmath}
\usepackage{amssymb}
\usepackage{mathabx}
\usepackage{titlesec}
\usepackage{fullpage}
\usepackage{tikz-cd}
\usepackage{rotating}
\usepackage{pdflscape}
\usepackage{multicol}
\usepackage{multirow}
\usepackage{diagbox}
\usepackage[left=.5in,right=.5in,top=.5in,bottom=.5in]{geometry}
\usepackage{enumitem}
\usepackage[colorlinks, hypertexnames=false]{hyperref}
\usepackage{bussproofs}
\usepackage[normalem]{ulem}

\setitemize{topsep=3pt,parsep=5pt,partopsep=0pt,label=,leftmargin=1.3pc}
\titleformat{\section}[runin]{\normalfont\bfseries}{\thesection}{0.5em}{}
\titlespacing{\section}{0pc}{5ex plus .1ex minus .2ex}{1pc}
\titleformat{\subsection}[runin]{\normalfont\bfseries}{\thesubsection}{0.7em}{}
\titlespacing{\subsection}{0pc}{2ex plus .1ex minus .2ex}{1pc}
\titleformat{\subsubsection}[runin]{\normalfont\bfseries}{\thesubsubsection}{0.7em}{}
\titlespacing{\subsubsection}{0pc}{2ex plus .1ex minus .2ex}{1pc}
\newcommand\eqn{\refstepcounter{equation}\tag{\theequation}}
\binoppenalty=\maxdimen
\relpenalty=\maxdimen
\newcommand{\ul}{\ulcorner}
\newcommand{\ur}{\urcorner}
\newcommand{\val}[1]{\ulcorner len1 \urcorner}
\newcommand{\caseref}[1]{\hyperref[#1]{\ref{#1}}}
\newcommand{\rot}{\rotatebox{90}}
\newcommand{\p}{\partial}
\newcommand{\todo}[1]{{\color{red}\textbf{TODO} #1}}
\newcommand{\red}{\color{red}}
\newcommand{\stl}{\mathbf{STL}}
\newcommand{\st}{\text{ST}}
\newcommand{\qst}{\mathbf{QST}}
\newcommand{\gstl}{\mathbf{GSTL}}
\newcommand{\gst}{\text{GST}}
\newcommand{\istl}{\mathbf{iSTL}}
\newcommand{\ist}{\text{iST}}
\newcommand{\igstl}{\mathbf{iGSTL}}
\newcommand{\igst}{\text{iGST}}
\newcommand{\sstl}{\mathbb{S}\mathbf{TL}}
\newcommand{\gsstl}{\mathbf{G}\mathbb{S}\mathbf{TL}}
\newcommand{\stt}{\mathbf{ST3}}
\newcommand{\qstt}{\mathbf{QST3}}
\newcommand{\gstt}{\text{GST3}}
\newcommand{\gsttl}{\mathbf{GST3L}}
\newcommand{\istt}{\text{iST3}}
\newcommand{\igstt}{\text{iGST3}}
\newcommand{\D}{\mathcal{D}}
\newcommand{\IH}{\text{IH}}
\newcommand{\qed}{{\begin{flushright}$\Box$\end{flushright}}}
\newtheorem{proposition}{}[section]
\EnableBpAbbreviations

\begin{document}
{\noindent
	v 2 \\
{\large\textbf{Decidability of $\ldl$}}
}
\\
\begin{multicols}{3}
  \begin{prooftree}
    \AXC{}
    \RightLabel{$Id ^p$}
    \UIC{$p \Rightarrow p$}
  \end{prooftree}
  \columnbreak
  \begin{prooftree}
    \AXC{}
    \RightLabel{$L \bot$}
    \UIC{$\uwave{\bot} \Rightarrow$}
  \end{prooftree}
  \columnbreak
  \begin{prooftree}
    \AXC{}
    \RightLabel{$R \top$}
    \UIC{$\Rightarrow \uwave{\top}$}
  \end{prooftree}
\end{multicols}

\begin{multicols}{2}
  \begin{prooftree}
    \AXC{$ \Gamma \Rightarrow \Delta$}
    \RightLabel{$LW$}
    \UIC{$ \Gamma, {\Sigma} \Rightarrow \Delta$}
  \end{prooftree}
  \columnbreak
  \begin{prooftree}
    \AXC{$ \Gamma \Rightarrow$}
    \RightLabel{$Rw$}
     \UIC{$\Gamma \Rightarrow {A}$}		
  \end{prooftree}
 \end{multicols}

\begin{multicols}{2}
  \begin{prooftree}
    \AXC{$\Gamma, \nabla^n A, \nabla^n B \Rightarrow \Delta$}
    \RightLabel{$L \wedge ^n$}
    \UIC{$\Gamma, \uwave{\nabla^n (A \wedge B)} \Rightarrow \Delta$}		
  \end{prooftree}
  \columnbreak
  \begin{prooftree}
    \AXC{$\Gamma \Rightarrow A$}
    \AXC{$\Gamma \Rightarrow B$}
    \RightLabel{$R \wedge$}
    \BIC{$ \Gamma \Rightarrow \uwave{A \wedge B}$}		
  \end{prooftree}
\end{multicols}

\begin{prooftree}
  \AXC{$ \Gamma, \nabla^n A \Rightarrow \Delta$}
  \AXC{$\Gamma, \nabla^n B \Rightarrow \Delta$}
  \RightLabel{$L \vee ^n$}
  \BIC{$ \Gamma, \uwave{\nabla^n (A \vee B)} \Rightarrow \Delta$}		
\end{prooftree}

\begin{multicols}{2}
  \columnbreak
  \begin{prooftree}
    \AXC{$\Gamma \Rightarrow A$}
    \RightLabel{$R \vee_1$}
    \UIC{$\Gamma \Rightarrow \uwave{A \vee B}$}		
  \end{prooftree}
  \columnbreak
  \begin{prooftree}
    \AXC{$\Gamma \Rightarrow B$}
    \RightLabel{$R \vee_2$}
    \UIC{$\Gamma \Rightarrow \uwave{A \vee B}$}		
  \end{prooftree}
\end{multicols}


\begin{prooftree}
  \AXC{$\Gamma, \nabla^{n+1} (A \rightarrow B) \Rightarrow \nabla^n A$}
  \AXC{$\Gamma, \nabla^{n+1} (A \rightarrow B), \nabla^n B \Rightarrow \Delta$}
  \RightLabel{$L \rightarrow ^n$}
  \BIC{$\Gamma, \uwave{\nabla^{n+1} (A \rightarrow B)} \Rightarrow \Delta$}		
\end{prooftree}

\begin{prooftree}
  \AXC{$\nabla \Gamma, A \Rightarrow B$}
  \RightLabel{$R \rightarrow$}
  \UIC{$\Gamma \Rightarrow \uwave{A \rightarrow B}$}		
\end{prooftree}

\begin{prooftree}
  \AXC{$\Gamma, \nabla^n (A \supset B) \Rightarrow \nabla^n A$}
  \AXC{$\Gamma, \nabla^n B \Rightarrow \Delta$}
  \RightLabel{$L \supset ^n$}
  \BIC{$\Gamma, \uwave{\nabla^n (A \supset B)} \Rightarrow \Delta$}		
\end{prooftree}

\begin{prooftree}
  \AXC{$\Gamma, A \Rightarrow B$}
  \RightLabel{$R \supset$}
  \UIC{$\Gamma \Rightarrow \uwave{A \supset B}$}		
\end{prooftree}

\begin{prooftree}
  \AXC{$\Gamma \Rightarrow \Delta$}
  \RightLabel{$N$}
  \UIC{$\nabla \Gamma \Rightarrow \nabla \Delta$}
\end{prooftree}

We refer to the rules $Id ^p$, $L \bot$ and $R \top$ as the \emph{axioms}, $L \wedge ^n$, $R \wedge$, $L \vee ^n$, $R \vee_1$, $R \vee_2$, $L \rightarrow ^n$, $R \rightarrow$, $L \supset ^n$ and $R \supset$ as the \emph{logical rules}, and $LW$ and $RW$ as the \emph{structural rules}.

\section{Weight}
\[
	w_\nabla(A) =
	\begin{cases}
		0 & ; A = p, \top, \bot \\
		w_\nabla(B) + 1 & ; A = \nabla B \\
		\max(w_\nabla(B), w_\nabla(C)) & ; A = B \Box C \quad (\Box \in \{ \wedge, \vee, \rightarrow, \supset \})
	\end{cases}
\]

\[
	w_\rightarrow(A) =
	\begin{cases}
		0 & ; A = p, \top, \bot \\
		w_\nabla(B) & ; A = \nabla B \\
		\max(w_\nabla(B), w_\nabla(C)) + 1 & ; A = B \rightarrow C \\
		\max(w_\nabla(B), w_\nabla(C)) & ; A = B \Box C \quad (\Box \in \{ \wedge, \vee, \supset \})
	\end{cases}
\]
In the following, let $\circ \in \{ \rightarrow, \nabla \}$.
\[ w_\circ(\Gamma) = \max\{w_\circ(A) \mid A \in \Gamma\} \]
\[ w_\circ(\Gamma \Rightarrow \Delta) = w_\circ(\Gamma, \Delta) \]
For any proof-tree $\D$, $\circ$-weight of $\D$, denoted by $w_\circ$, is defined to be the maximum $w_\circ$ among all its sequents. So if $\D$ ends with a rule with conclusion $\Gamma \Rightarrow \Delta$ and premises $\{ \D_i \}$, then
\[ w_\circ(\D) = \max\{w_\circ(\Gamma \Rightarrow \Delta), w_\circ(\D_i)\} \]

\subsection{Remark}\quad
\begin{enumerate}
	\item $w_\circ(\nabla^n A) \le w_\circ(\nabla^n (A \Box B))$, where $\Box \in \{\wedge, \vee, \rightarrow, \supset\}$
	\item $w_\circ(\Gamma) \le w_\circ(\Gamma, A)$
	\item $\max(w_\circ(A, B)) = \max(w_\circ(A), w_\circ(B))$
\end{enumerate}

\section{Upper-bound for $w_\nabla$} For any sequent $\Gamma \Rightarrow \Delta$, if $\vdash \Gamma \Rightarrow \Delta$ then it is provable by a proof-tree $\D$ such that $w_\nabla(\D) \le w_\nabla(\Gamma \Rightarrow \Delta) + w_\rightarrow(\Delta)$.

\emph{Proof.} By induction on the proof of $\Gamma \Rightarrow \Delta$. The cases for axioms are trivial. In all cases, where the last rule is $R$, take the proof-tree from the induction hypothesis, called $\D'$ (and $\D''$, for binary rules), and observe that applying $R$ will not increase the $\nabla$-weight of the resulting proof-tree, which we call $\D$, more than the desired upper-bound. The claim is clear when $R$ is $LW$ or $Rw$. Such is the case for $R \wedge$, $R \vee_1$, $R \vee_2$, $R \supset$, $L \wedge^n$, $L \vee^n$ or $L \supset^n$, since the $\nabla$-weight does not increase in the conclusion. For example, let $R$ be $R \wedge$. We have
\[ w_\nabla(\D') \le w_\nabla(\Gamma, A) + w_\rightarrow(A) \]
\[ w_\nabla(\D'') \le w_\nabla(\Gamma, B) + w_\rightarrow(B) \]
Construct $\D$ by applying $R \wedge$ on $\D'$ and $\D''$. By definition
\[ w_\nabla(\D) = \max\{w_\nabla(\Gamma, A \wedge B), w_\nabla(\D'), w_\nabla(\D'')\} \]
So we have three cases. If $w_\nabla(\D) \le w_\nabla(\Gamma, A \wedge B)$ the claim is clear. If $w_\nabla(\D) \le w_\nabla(\D')$ then we have
\[ w_\nabla(\D) \le w_\nabla(\Gamma, A) + w_\rightarrow(A) \]
By Remark we have
\[ w_\nabla(\D) \le w_\nabla(\Gamma, A \wedge B) + w_\rightarrow(A \wedge B) \]
The other case where $w_\nabla(\D) \le w_\nabla(\D'')$ is similar.

Let $R$ be $L \supset^n$. We have
\[ w_\nabla(\D') \le w_\nabla(\Gamma, \nabla^n (A \supset B), \nabla^n A) + w_\rightarrow(\nabla^n A) \]
\[ w_\nabla(\D'') \le w_\nabla(\Gamma, \nabla^n B, \Delta) + w_\rightarrow(\Delta) \]
Construct $\D$ by applying $R \supset^n$ on $\D'$ and $\D''$. By definition
\[ w_\nabla(\D) = \max\{w_\nabla(\Gamma, \nabla^n (A \supset B), \Delta), w_\nabla(\D'), w_\nabla(\D'')\} \]
The first case is clear. If $w_\nabla(\D) \le w_\nabla(\D')$ then we have
\[ w_\nabla(\D'') \le w_\nabla(\D') \]
\[ w_\nabla(\Gamma, \nabla^n (A \supset B), \Delta) \le w_\nabla(\Gamma, \nabla^n (A \supset B), \nabla^n A) + w_\rightarrow(\nabla^n A) \]
\[ w_\nabla(\D) \le w_\nabla(\Gamma, \nabla^n (A \supset B), \nabla^n A) + w_\rightarrow(\nabla^n A) \]
We want to prove
\[ w_\nabla(D) \le w_\nabla(\Gamma, \nabla^n (A \supset B), \Delta) + w_\rightarrow(\Delta) \]
\end{document} 
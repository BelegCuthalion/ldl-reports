\documentclass[a4paper, 12pt]{paper}
\usepackage{amsmath}
\usepackage{amssymb}
\usepackage{mathabx}
\usepackage{titlesec}
\usepackage{fullpage}
\usepackage{tikz-cd}
\usepackage{rotating}
\usepackage{pdflscape}
\usepackage{multicol}
\usepackage{multirow}
\usepackage{diagbox}
\usepackage[left=.5in,right=.5in,top=.5in,bottom=.5in]{geometry}
\usepackage{enumitem}
\usepackage[colorlinks]{hyperref}
\usepackage{bussproofs}

\setitemize{topsep=3pt,parsep=5pt,partopsep=0pt,label=,leftmargin=1.3pc}
\titleformat{\section}[runin]{\normalfont\bfseries}{\thesection}{0.5em}{}
\titlespacing{\section}{0pc}{5ex plus .1ex minus .2ex}{1pc}
\titleformat{\subsection}[runin]{\normalfont\bfseries}{\thesubsection}{0.7em}{}
\titlespacing{\subsection}{0pc}{2ex plus .1ex minus .2ex}{1pc}
\titleformat{\subsubsection}[runin]{\normalfont\bfseries}{\thesubsubsection}{0.7em}{}
\titlespacing{\subsubsection}{0pc}{2ex plus .1ex minus .2ex}{1pc}
\newcommand\eqn{\refstepcounter{equation}\tag{\theequation}}
\binoppenalty=\maxdimen
\relpenalty=\maxdimen
\newcommand{\ul}{\ulcorner}
\newcommand{\ur}{\urcorner}
\newcommand{\val}[1]{\ulcorner len1 \urcorner}
\newcommand{\caseref}[1]{\hyperref[#1]{\ref{#1}}}
\newcommand{\rot}{\rotatebox{90}}
\newcommand{\p}{\partial}
\newcommand{\todo}[1]{{\color{red}\textbf{TODO} #1}}
\newcommand{\red}{\color{red}}
\newcommand{\stl}{\mathbf{STL}}
\newcommand{\gstl}{\mathbf{GSTL}}
\newcommand{\istl}{\mathbf{iSTL}}
\newcommand{\igstl}{\mathbf{iGSTL}}
\newcommand{\sstl}{\mathbb{S}\mathbf{TL}}
\newtheorem{proposition}{}[section]
\EnableBpAbbreviations

\begin{document}
{\noindent
	v 1.2 \\
  \\
{\Huge\textbf{A Contraction-free System for STL}}
}
\\
\part*{Definitions}
\setcounter{section}{-1}
\section{Notation} In the following, a formula is a free construction of operators $\bot$, $\top$, $\nabla$, $\vee$, $\wedge$ and $\rightarrow$, which have arities $0$, $0$, $1$, $2$, $2$ and $2$, respectively, over a countable set of atoms. $\Gamma$, $\Sigma$ and $\Pi$ are names for finite multi-sets of formulas, $\Delta$ for a sub-singleton of some formula, $A$, $B$ and $C$ for formulas, $p$ for atoms and $n$, $l$, $r$ and $k$ for natural numbers. We denote by $P$ the set of all atoms, and by $P(A)$ the set of those that occur in the formula $A$.

We will write $A$ for the singleton $\{A\}$ where ever it is inferable from the context.
``$,$'' is the multi-set union. We will also use the following short-hands.
\begin{flushleft}
  $ A^1 = \{ A \} $ \\
  $ A^{n+1} = A^n, A $ \\
  $ \Gamma^1 = \Gamma $ \\
  $ \Gamma^{n+1} = \Gamma^n, \Gamma $ \\
  $ \nabla^1 A = \nabla A $ \\
  $ \nabla^{n+1} A = \nabla (\nabla^n A) $ \\
  $ \nabla^n \Gamma = \{ \nabla^n A \mid A \in \Gamma \} $ \\
  $ P(\Gamma) = \bigcup_{A \in \Gamma} P(A) $
\end{flushleft}


A sequent $\Gamma \Rightarrow \Delta$ is a binary relation between $\Gamma$, a multi-set of formulas called \emph{the antecedent} or \emph{the left-side}, and $\Delta$, a sub-singleton of some formula called \emph{the succedent} or \emph{the right-side}.


We will construct trees, called \emph{proof-trees}, using a set of recursive rules, called a \emph{deductive system}, or simply \emph{system}.

A rule $R$ is expressed as a relation between a set of $n$ sequents $\{ \Gamma_i \Rightarrow \Delta_i \mid 1 \leq i \leq n \}$ called \emph{premises} and a sequent $\Gamma \Rightarrow \Delta$ called \emph{conclusion}, and is written as follows.
\begin{prooftree}
  \AXC{$\Gamma_1 \Rightarrow \Delta_1$}
  \AXC{$\dots$}
  \AXC{$\Gamma_n \Rightarrow \Delta_n$}
  \RightLabel{$R$}
  \TIC{$\Gamma \Rightarrow \Delta$}
\end{prooftree}

We will designate a specific formula in the conclusion of some rules, called \emph{the principal formula} of that rule, which will be indicated by writing it \uwave{underlined}.

A proof-tree in a system is made up of instances of rules of that system. So in a system with the rule $R$, we can construct a proof-tree with root $\Gamma \Rightarrow \Delta$, if we have already constructed $n$ proof-trees with roots $\Gamma_i \Rightarrow \Delta_i$ for all $1 \leq i \leq n$.



When a proof-tree, with $\Gamma \Rightarrow \Delta$ as root, is constructed using a system $S$, we say that $S$ proves $\Gamma \Rightarrow \Delta$, and write it as $S \vdash \Gamma \Rightarrow \Delta$.



We will name proof-trees by $\D$, $\D'$ and so on, unless otherwise stated. We will name subtrees of $\D$ by $\D_0$, $\D_1$ and so on. We will frequently use induction to prove existence of some proof-tree, assuming existence of some other proof-tree $\D$. In such situations we will use notation $IH(\D)$ to denote the proof-tree that we will get from induction hypothesis.

We will write $S \vdash_h \Gamma \Rightarrow \Delta$ to indicate the existence of a proof-tree of height $h$ for $\Gamma \Rightarrow \Delta$ in system $S$. We will also write $h(\D)$ for the height of a proof-tree $\D$.

We write $S \vdash^r A$ when we want to indicate that a system $S$ proves a formula $A$, with a proof-tree of rank $r$ (defined below).


\section{cut} \quad \\

\begin{center}
  \begin{prooftree}
    \AXC{$ \Gamma \Rightarrow A$}
    \AXC{$\Sigma, A \Rightarrow \Delta$}
    \RightLabel{$cut$}
    \BIC{$\Gamma, \Sigma \Rightarrow \Delta$}
  \end{prooftree}
\end{center}

\begin{center}
  \begin{prooftree}
    \AXC{$ \Gamma \Rightarrow A$}
    \AXC{$\Sigma, \nabla^n A \Rightarrow \Delta$}
    \RightLabel{$\nabla cut$}
    \BIC{$\nabla^n \Gamma, \Sigma \Rightarrow \Delta$}
  \end{prooftree}
\end{center}

Notice that $\nabla cut$ generalizes $cut$.

$A$ is called \emph{the cut-formula} of a $cut$ or $\nabla cut$ instance.

\emph{Rank} of a formula $A$ is defined as
\[ \rho(A) = \begin{cases}
1 & \quad ; A \in P \cup \{ \bot, \top \} \\
\rho(B) & \quad ; A = \nabla B \\
max(\rho(B), \rho(C)) + 1 & \quad ; A = B \circ C \quad (\circ \in \{ \land, \lor, \rightarrow \})
\end{cases} \]
Notice that $\nabla$ does not increase the rank of a formula.

We also define rank for rule instances and proof-trees as follows. Rank of an instance of the $\nabla cut$ rule with cut-formula $A$ is defined to be the rank of $A$. Rank of any other rule instance is $0$.
Rank of a proof-tree $\D$, written as $\rho(\D)$, is the maximum rank of all of its rule instances.


\section{ST} \quad \\

 \begin{multicols}{3}
  \begin{prooftree}
    \AXC{}
    \RightLabel{$Id$}
    \UIC{$A \Rightarrow A$}
  \end{prooftree}
  \columnbreak
  \begin{prooftree}
    \AXC{}
    \RightLabel{$L \bot$}
    \UIC{$ \bot \Rightarrow $}		
  \end{prooftree}
  \columnbreak
  \begin{prooftree}
    \AXC{}
    \RightLabel{$R \top$}
    \UIC{$ \Rightarrow \top$}
  \end{prooftree}
\end{multicols}

\begin{multicols}{3}
  \begin{prooftree}
    \AXC{$ \Gamma, A \Rightarrow \Delta$}
    \RightLabel{$L \wedge_1$}
    \UIC{$ \Gamma, \uwave{A \wedge B} \Rightarrow \Delta$}		
  \end{prooftree}
  \columnbreak
  \begin{prooftree}
    \AXC{$ \Gamma, B \Rightarrow \Delta$}
    \RightLabel{$L \wedge_2$}
    \UIC{$\Gamma, \uwave{A \wedge B} \Rightarrow \Delta$}		
  \end{prooftree}
  \columnbreak
  \begin{prooftree}
    \AXC{$\Gamma \Rightarrow A$}
    \AXC{$\Gamma \Rightarrow B$}
    \RightLabel{$R \wedge$}
    \BIC{$ \Gamma \Rightarrow \uwave{A \wedge B}$}
  \end{prooftree}
\end{multicols}

\begin{multicols}{3}
  \begin{prooftree}
    \AXC{$ \Gamma, A \Rightarrow \Delta$}
    \AXC{$\Gamma, B \Rightarrow \Delta$}
    \RightLabel{$L \vee$}
    \BIC{$ \Gamma, \uwave{A \vee B} \Rightarrow \Delta$}		
  \end{prooftree}
  \columnbreak
  \begin{prooftree}
    \AXC{$\Gamma \Rightarrow A$}
    \RightLabel{$R \vee_1$}
    \UIC{$\Gamma \Rightarrow \uwave{A \vee B}$}		
  \end{prooftree}
  \columnbreak
  \begin{prooftree}
    \AXC{$\Gamma \Rightarrow B$}
    \RightLabel{$R \vee_2$}
    \UIC{$\Gamma \Rightarrow \uwave{A \vee B}$}		
  \end{prooftree}
\end{multicols}

 \begin{multicols}{2}
  \begin{prooftree}
    \AXC{$\Gamma \Rightarrow A$}
    \AXC{$\Gamma, B \Rightarrow \Delta$}
    \RightLabel{$L \rightarrow$}
    \BIC{$\Gamma, \uwave{\nabla (A \rightarrow B)} \Rightarrow \Delta$}		
  \end{prooftree}
  \columnbreak
  \begin{prooftree}
    \AXC{$\nabla \Gamma, A \Rightarrow B$}
    \RightLabel{$R \rightarrow$}
    \UIC{$\Gamma \Rightarrow \uwave{A \rightarrow B}$}		
  \end{prooftree}
\end{multicols}

\begin{multicols}{2}
 \begin{prooftree}
   \AXC{$ \Gamma \Rightarrow \Delta$}
   \RightLabel{$L w$}
   \UIC{$ \Gamma, \uwave{A} \Rightarrow \Delta$}
 \end{prooftree}
 \columnbreak
 \begin{prooftree}
   \AXC{$ \Gamma \Rightarrow$}
   \RightLabel{$R w$}
    \UIC{$\Gamma \Rightarrow \uwave{A}$}		
 \end{prooftree}
\end{multicols}

\begin{multicols}{1}
 \begin{prooftree}
   \AXC{$ \Gamma, A, A \Rightarrow \Delta$}
   \RightLabel{$Lc$}
   \UIC{$\Gamma, \uwave{A} \Rightarrow \Delta$}		
 \end{prooftree}
\end{multicols}

\begin{prooftree}
  \AXC{$\Gamma \Rightarrow \Delta$}
  \RightLabel{$N$}
  \UIC{$\nabla \Gamma \Rightarrow \nabla \Delta$}
\end{prooftree}



By $\st \vdash \Gamma \Rightarrow \Delta$ we mean that the sequent $\Gamma \Rightarrow \Delta$ provable in this system.

By $\st^+ \vdash \Gamma \Rightarrow \Delta$ we mean that the sequent $\Gamma \Rightarrow \Delta$ provable in the above system, augmented with the $cut$ rule. 
We denote the \emph{spacetime logic} by $\stl$, which is the set of all sequents provable in $\st^+$.



\section{ST3} \quad \\

 \begin{multicols}{3}
  \begin{prooftree}
    \AXC{}
    \RightLabel{$Id$}
    \UIC{$\Gamma, p \Rightarrow p$}
  \end{prooftree}
  \columnbreak
  \begin{prooftree}
    \AXC{}
    \RightLabel{$L \bot$}
    \UIC{$\Gamma, \bot \Rightarrow \Delta$}
  \end{prooftree}
  \columnbreak
  \begin{prooftree}
    \AXC{}
    \RightLabel{$R \top$}
    \UIC{$\Gamma \Rightarrow \top$}
  \end{prooftree}
\end{multicols}

\begin{multicols}{2}
  \begin{prooftree}
    \AXC{$\Gamma, A, B \Rightarrow \Delta$}
    \RightLabel{$L \wedge$}
    \UIC{$\Gamma, \uwave{A \wedge B} \Rightarrow \Delta$}		
  \end{prooftree}
  \columnbreak
  \begin{prooftree}
    \AXC{$\Gamma \Rightarrow A$}
    \AXC{$\Gamma \Rightarrow B$}
    \RightLabel{$R \wedge$}
    \BIC{$ \Gamma \Rightarrow \uwave{A \wedge B}$}		
  \end{prooftree}
\end{multicols}

\begin{multicols}{3}
  \begin{prooftree}
    \AXC{$ \Gamma, A \Rightarrow \Delta$}
    \AXC{$\Gamma, B \Rightarrow \Delta$}
    \RightLabel{$L \vee$}
    \BIC{$ \Gamma, \uwave{A \vee B} \Rightarrow \Delta$}		
  \end{prooftree}
  \columnbreak
  \begin{prooftree}
    \AXC{$\Gamma \Rightarrow A$}
    \RightLabel{$R \vee_1$}
    \UIC{$\Gamma \Rightarrow \uwave{A \vee B}$}		
  \end{prooftree}
  \columnbreak
  \begin{prooftree}
    \AXC{$\Gamma \Rightarrow B$}
    \RightLabel{$R \vee_2$}
    \UIC{$\Gamma \Rightarrow \uwave{A \vee B}$}		
  \end{prooftree}
\end{multicols}

 \begin{multicols}{2}
  \begin{prooftree}
    \AXC{$\Gamma, \nabla (A \rightarrow B) \Rightarrow A$}
    \AXC{$\Gamma, B \Rightarrow \Delta$}
    \RightLabel{$L \rightarrow$}
    \BIC{$\Gamma, \uwave{\nabla (A \rightarrow B)} \Rightarrow \Delta$}		
  \end{prooftree}
  \columnbreak
  \begin{prooftree}
    \AXC{$\nabla \Gamma, A \Rightarrow B$}
    \RightLabel{$R \rightarrow$}
    \UIC{$\Gamma \Rightarrow \uwave{A \rightarrow B}$}		
  \end{prooftree}
\end{multicols}

\begin{multicols}{2}
  \begin{prooftree}
    \AXC{$ \Gamma \Rightarrow \Delta$}
    \RightLabel{$L w$}
    \UIC{$ \Gamma, \uwave{\Sigma} \Rightarrow \Delta$}
  \end{prooftree}
  \columnbreak
  \begin{prooftree}
    \AXC{$ \Gamma \Rightarrow$}
    \RightLabel{$R w$}
     \UIC{$\Gamma \Rightarrow \uwave{A}$}		
  \end{prooftree}
 \end{multicols}

\begin{prooftree}
  \AXC{$\Gamma \Rightarrow \Delta$}
  \RightLabel{$N$}
  \UIC{$\nabla \Gamma \Rightarrow \nabla \Delta$}
\end{prooftree}



Notice that there could be more than one principal-formula in $Lw$.

By $\stt \vdash \Gamma \Rightarrow \Delta$ we mean that the sequent $\Gamma \Rightarrow \Delta$ provable in this system.

By $\stt^+ \vdash \Gamma \Rightarrow \Delta$ we mean that the sequent $\Gamma \Rightarrow \Delta$ provable in the above system, augmented with the $cut$ rule.



\section{GST3} \quad \\

 \begin{multicols}{3}
  \begin{prooftree}
    \AXC{}
    \RightLabel{$Id$}
    \UIC{$\Gamma, p \Rightarrow p$}
  \end{prooftree}
  \columnbreak
  \begin{prooftree}
    \AXC{}
    \RightLabel{$L \bot$}
    \UIC{$\Gamma, \bot \Rightarrow \Delta$}
  \end{prooftree}
  \columnbreak
  \begin{prooftree}
    \AXC{}
    \RightLabel{$R \top$}
    \UIC{$\Gamma \Rightarrow \top$}
  \end{prooftree}
\end{multicols}

\begin{multicols}{2}
  \begin{prooftree}
    \AXC{$\Gamma, \nabla^n A, \nabla^n B \Rightarrow \Delta$}
    \RightLabel{$L \wedge$}
    \UIC{$\Gamma, \uwave{\nabla^n (A \wedge B)} \Rightarrow \Delta$}		
  \end{prooftree}
  \columnbreak
  \begin{prooftree}
    \AXC{$\Gamma \Rightarrow A$}
    \AXC{$\Gamma \Rightarrow B$}
    \RightLabel{$R \wedge$}
    \BIC{$ \Gamma \Rightarrow \uwave{A \wedge B}$}		
  \end{prooftree}
\end{multicols}

\begin{multicols}{3}
  \begin{prooftree}
    \AXC{$ \Gamma, \nabla^n A \Rightarrow \Delta$}
    \AXC{$\Gamma, \nabla^n B \Rightarrow \Delta$}
    \RightLabel{$L \vee$}
    \BIC{$ \Gamma, \uwave{\nabla^n (A \vee B)} \Rightarrow \Delta$}		
  \end{prooftree}
  \columnbreak
  \begin{prooftree}
    \AXC{$\Gamma \Rightarrow A$}
    \RightLabel{$R \vee_1$}
    \UIC{$\Gamma \Rightarrow \uwave{A \vee B}$}		
  \end{prooftree}
  \columnbreak
  \begin{prooftree}
    \AXC{$\Gamma \Rightarrow B$}
    \RightLabel{$R \vee_2$}
    \UIC{$\Gamma \Rightarrow \uwave{A \vee B}$}		
  \end{prooftree}
\end{multicols}

 \begin{multicols}{2}
  \begin{prooftree}
    \AXC{$\Gamma, \nabla^{n+1} (A \rightarrow B) \Rightarrow \nabla^n A$}
    \AXC{$\Gamma, \nabla^{n+1} (A \rightarrow B), \nabla^n B \Rightarrow \Delta$}
    \RightLabel{$L \rightarrow$}
    \BIC{$\Gamma, \uwave{\nabla^{n+1} (A \rightarrow B)} \Rightarrow \Delta$}		
  \end{prooftree}
  \columnbreak
  \begin{prooftree}
    \AXC{$\nabla \Gamma, A \Rightarrow B$}
    \RightLabel{$R \rightarrow$}
    \UIC{$\Gamma \Rightarrow \uwave{A \rightarrow B}$}		
  \end{prooftree}
\end{multicols}

\begin{multicols}{2}
  \begin{prooftree}
    \AXC{$ \Gamma \Rightarrow \Delta$}
    \RightLabel{$L w$}
    \UIC{$ \Gamma, \uwave{\Sigma} \Rightarrow \Delta$}
  \end{prooftree}
  \columnbreak
  \begin{prooftree}
    \AXC{$ \Gamma \Rightarrow$}
    \RightLabel{$R w$}
     \UIC{$\Gamma \Rightarrow \uwave{A}$}		
  \end{prooftree}
 \end{multicols}

\begin{prooftree}
  \AXC{$\Gamma \Rightarrow \Delta$}
  \RightLabel{$N$}
  \UIC{$\nabla \Gamma \Rightarrow \nabla \Delta$}
\end{prooftree}


Notice that all $\gstt$ rules generalize $\stt$ rules.

Also notice that there could be more than one principal-formula in $Lw$.

By $\gstt \vdash \Gamma \Rightarrow \Delta$ we mean that the sequent $\Gamma \Rightarrow \Delta$ provable in this system.

By $\gstt^+ \vdash \Gamma \Rightarrow \Delta$ we mean that the sequent $\Gamma \Rightarrow \Delta$ provable in the above system, augmented with the $\nabla cut$ rule.

We denote the logic of $\gstt$, which is the set of all sequents provable in $\gstt$, by $\gsttl$.


\section{Conventions} We will refer to $Id$, $L \bot$, $R \top$ as \emph{axioms}, $L \wedge$, $L \wedge_1$, $L \wedge_2$, $L \vee$, $R \vee_1$, $R \vee_2$, $L \rightarrow$, $R \rightarrow$ as \emph{logical rules}, $Lw$, $Rw$ and $Lc$ as \emph{structural rules} and $N$ as \emph{the modal rule}.


\pagebreak

\part*{Theorems}
\setcounter{section}{0}


\section{Theorem}\label{thm:stt-id-form} $(Id)$ $\stt \vdash \Gamma, A \Rightarrow A$. \hyperref[pr:stt-id-form]{\emph{Proof.}}


\section{Lemma}\label{lem:stt-inversion}
\begin{enumerate}
  \item If $\stt \vdash_h \Gamma, \nabla^n (A \wedge B) \Rightarrow \Delta$ then $\stt \vdash_h \Gamma, \nabla^n A, \nabla^n B \Rightarrow \Delta$.
  \item If $\stt \vdash_h \Gamma, \nabla^n (A \vee B) \Rightarrow \Delta$ then $\stt \vdash_h \Gamma, \nabla^n A \Rightarrow \Delta$ and $\stt \vdash_h \Gamma, \nabla^n B \Rightarrow \Delta$.
  \item If $\stt \vdash_h \Gamma \Rightarrow A \wedge B$ then $\stt \vdash_h \Gamma \Rightarrow A$ and $\stt \vdash_h \Gamma \Rightarrow B$.
\end{enumerate}
\hyperref[pr:stt-inversion]{\emph{Proof.}}


\section{Theorem}\label{thm:stt-lc-elim} $(Lc)$ If $\stt \vdash_h \Gamma, A^2 \Rightarrow \Delta$ then $\stt \vdash_h \Gamma, A \Rightarrow \Delta$.
\hyperref[pr:stt-lc-elim]{\emph{Proof.}}



\section{Theorem}\label{thm:stt-eq-st} $\stt \vdash \Gamma \Rightarrow \Delta$ iff $\st \vdash \Gamma \Rightarrow \Delta$.
\hyperref[pr:stt-eq-st]{\emph{Proof.}}



\section{Theorem}\label{thm:sttp-eq-stp} $\stt^+ \vdash \Gamma \Rightarrow \Delta$ iff $\st^+ \vdash \Gamma \Rightarrow \Delta$.
\hyperref[pr:sttp-eq-stp]{\emph{Proof.}}



\section{Lemma}\label{lem:l-nabla-dist}
\begin{enumerate}
  \item $\stt^+ \vdash \nabla^n (A \vee B) \Rightarrow \nabla^n A \vee \nabla^n B$.

  \item $\stt^+ \vdash \nabla^n (A \wedge B) \Rightarrow \nabla^n A \wedge \nabla^n B$. 

  \item $\stt^+ \vdash \nabla^n (A \rightarrow B) \Rightarrow \nabla^n A \rightarrow \nabla^n B$.
\end{enumerate}
\hyperref[pr:l-nabla-dist]{\emph{Proof.}}



\section{Theorem}\label{thm:sttp-eq-gsttp} $\stt^+ \vdash \Gamma \Rightarrow \Delta$ iff $\gstt^+ \vdash \Gamma \Rightarrow \Delta$. \hyperref[pr:sttp-eq-gsttp]{\emph{Proof.}}



\section{Theorem}\label{lem:gstt-top-redundant}
If $\gstt \vdash^r \Gamma , \nabla^n \top \Rightarrow \Delta$ then $\gstt \vdash^r \Gamma \Rightarrow \Delta$.
\hyperref[pr:gstt-top-redundant]{\emph{Proof.}}



\section{Theorem}\label{thm:gstt-cut-reduction}
If $\gstt \vdash^{r_0} \Gamma \Rightarrow A$ and $\gstt \vdash^{r_1} \Sigma , \nabla^n A \Rightarrow \Delta$, where $r_0, r_1 < \rho(A)$, then $\gstt \vdash^{r_2} \nabla^n \Gamma, \Sigma \Rightarrow \Delta$ and $r_2 < \rho(A)$.\hyperref[pr:gstt-cut-reduction]{\emph{Proof.}}



\section{Theorem}\label{thm:gstt-eq-gsttp} $\gstt \vdash \Gamma \Rightarrow \Delta$ iff $\gstt^+ \vdash \Gamma \Rightarrow \Delta$. \hyperref[pr:gstt-eq-gsttp]{\emph{Proof.}}



\section{Corollary}\label{thm:gstt-eq-stp} $\gstt \vdash \Gamma \Rightarrow \Delta$ iff $\st^+ \vdash \Gamma \Rightarrow \Delta$. \hyperref[pr:gstt-eq-stp]{\emph{Proof.}}



\section{Corollary} $\stl = \gsttl$, thus, we have a cut-free system for $\stl$.


\pagebreak

\part*{Proofs}
\setcounter{section}{0}

\section{Proof of \ref{thm:stt-id-form}}\label{pr:stt-id-form}
By induction on the structure of $A$. Let $A$ have any of the following forms.

($p$) $Id$ proves $\Gamma, p \Rightarrow p$.

($\bot$) $L \bot$ proves $\Gamma, \bot \Rightarrow \bot$.

($\top$) $R \top$ proves $\Gamma, \top \Rightarrow \top$.

($B \wedge C$) By IH we have $\stt \vdash B \Rightarrow B$ and $\stt \vdash C \Rightarrow C$. By $Lw$ we have $\stt \vdash B, C \Rightarrow B$ and $\stt \vdash B, C \Rightarrow C$. By $R \wedge$ we have $\stt \vdash B, C \Rightarrow B \wedge C$. By $L \wedge$ we have $\stt \vdash B \wedge C \Rightarrow B \wedge C$.
\begin{prooftree}
  \AXC{IH}
  \noLine
  \UIC{$\Gamma, B \Rightarrow B$}
  \RightLabel{$Lw$}
  \UIC{$\Gamma, B, C \Rightarrow B$}
  \AXC{IH}
  \noLine
  \UIC{$\Gamma, C \Rightarrow C$}
  \RightLabel{$Lw$}
  \UIC{$\Gamma, B, C \Rightarrow C$}
  \RightLabel{$R \wedge$}
  \BIC{$\Gamma, B, C \Rightarrow B \wedge C$}
  \RightLabel{$L \wedge$}
  \UIC{$\Gamma, B \wedge C \Rightarrow B \wedge C$}
\end{prooftree}

($B \vee C$)
\begin{prooftree}
  \AXC{IH}
  \noLine
  \UIC{$\Gamma, B \Rightarrow B$}
  \RightLabel{$R \vee$}
  \UIC{$\Gamma, B \Rightarrow B \vee C$}
  \AXC{IH}
  \noLine
  \UIC{$\Gamma, C \Rightarrow C$}
  \RightLabel{$R \vee$}
  \UIC{$\Gamma, C \Rightarrow B \vee C$}
  \RightLabel{$L \vee$}
  \BIC{$\Gamma, B \vee C \Rightarrow B \vee C$}
\end{prooftree}

($B \rightarrow C$)
\begin{prooftree}
  \AXC{IH} \noLine
  \UIC{$\nabla \Gamma, \nabla (B \rightarrow C), B \Rightarrow B$}
  \AXC{IH} \noLine
  \UIC{$\nabla \Gamma, \nabla (B \rightarrow C), C \Rightarrow C$}
  \RightLabel{$L \rightarrow$}
  \BIC{$\nabla \Gamma, \nabla (B \rightarrow C), B \Rightarrow C$}
  \RightLabel{$R \rightarrow$}
  \UIC{$\Gamma, B \rightarrow C \Rightarrow B \rightarrow C$}
\end{prooftree}

($\nabla B$)
\begin{prooftree}
  \AXC{IH} \noLine
  \UIC{$B \Rightarrow B$}
  \RightLabel{$N$}
  \UIC{$\nabla B \Rightarrow \nabla B$}
  \RightLabel{$Lw$}
  \UIC{$\Gamma, \nabla B \Rightarrow \nabla B$}
\end{prooftree}

\qed



\section{Proof of \ref{lem:stt-inversion}}\label{pr:stt-inversion} We will use induction on the assumed proof-tree $\D$ for $\Gamma, \nabla^n (A \wedge B) \Rightarrow \Delta$, $\Gamma, \nabla^n (A \vee B) \Rightarrow \Delta$ and $\Gamma \Rightarrow A \wedge B$ for parts $1$, $2$ and $3$ of Lemma, repsectively, where it ends with a rule $R$. Notice that our construction does not increase the proof size.

Cases where $R$ is an axiom are trivial. In cases where $R$ is $L \wedge$, $L \vee$ and $R \wedge$ in parts $1$, $2$ and $3$, respectively, and $A \wedge B$ and $A \vee B$ are principal, which implies $n = 0$, the subtree(s) of $\D$ would prove the desired sequent. In all other cases just commute $R$ with $IH(D_0)$ (and $IH(D_1)$ if $R$ has two premises).




\section{Proof of \ref{thm:stt-lc-elim}}\label{pr:stt-lc-elim}
By induction on the proof for $\Gamma, A^2 \Rightarrow \Delta$, which we call $\D$ and ends with rule $R$.
Cases for axioms are trivial. All cases where none of occurrences of $A$ are principal are handled by commuting $R$ with $IH(D_i)$. Now suppose $R$ is either of the following and $A$ is principal in $R$.
\begin{itemize}
  \item[$(L \wedge)$] Let $A = A \wedge B$.
  \begin{prooftree}
    \AXC{$\D_0$} \noLine
    \UIC{$\Gamma, A, B, A \wedge B \Rightarrow \Delta$}
    \RightLabel{$L \wedge$}
    \UIC{$\Gamma, (A \wedge B)^2 \Rightarrow \Delta$}		
  \end{prooftree}
  Then by Lemma \ref{lem:stt-inversion} we have $\Gamma, A^2, B^2 \Rightarrow \Delta$. Using induction hypothesis twice, we would have $\Gamma, A, B \Rightarrow \Delta$. $L \wedge$ will result in the desired sequent.

  \item[$(L \vee)$] Let $A = A \vee B$.
  \begin{prooftree}
    \AXC{$\D_0$} \noLine
    \UIC{$ \Gamma, A, A \vee B \Rightarrow \Delta$}
    \AXC{$\D_1$} \noLine
    \UIC{$\Gamma, B, A \vee B \Rightarrow \Delta$}
    \RightLabel{$L \vee$}
    \BIC{$ \Gamma, (A \vee B)^2 \Rightarrow \Delta$}		
  \end{prooftree}
  By Lemma \ref{lem:stt-inversion} for $\D_0$ and $\D_1$ we have $\Gamma, A^2 \Rightarrow \Delta$ and $\Gamma, B^2 \Rightarrow \Delta$. By induction hypothesis we have $\Gamma, A \Rightarrow \Delta$ and $\Gamma, B \Rightarrow \Delta$. By $L \vee$ we have the desired sequent.

  \item[$(L \rightarrow)$] Let $A = \nabla (A \rightarrow B)$.
  \begin{prooftree}
    \AXC{$\D_0$} \noLine
    \UIC{$\Gamma, (\nabla (A \rightarrow B))^2 \Rightarrow A$}
    \AXC{$\D_1$} \noLine
    \UIC{$\Gamma, B, (\nabla (A \rightarrow B))^2 \Rightarrow \Delta$}
    \RightLabel{$L \rightarrow$}
    \BIC{$\Gamma, (\nabla (A \rightarrow B))^2 \Rightarrow \Delta$}
  \end{prooftree}
  By induction hypothesis we have $\Gamma, \nabla (A \rightarrow B) \Rightarrow A$ and $\Gamma, B, \nabla (A \rightarrow B) \Rightarrow \Delta$. By $L \rightarrow$ we have the desired sequent.
\end{itemize}



\section{Proof of \ref{thm:stt-eq-st}}\label{pr:stt-eq-st}\quad \\
($\Rightarrow$) All the rules in $\stt$ are also in $\st$, except for $L \wedge$ and $L \rightarrow$, which can be easily imitated in $\st$ using the same logical rules and structural rules in $\st$. \\
($\Leftarrow$) All the rules in $\st$ are also in $\stt$, except for $Lc$, which can be imitated in $\stt$ using Theorem \ref{thm:stt-lc-elim}.



\section{Proof of \ref{thm:sttp-eq-stp}}\label{pr:sttp-eq-stp}\quad \\
Similar to \ref{pr:stt-eq-st}.\\
($\Rightarrow$) By induction on the structure of $\D$, the proof-tree for $\Gamma \Rightarrow \Delta$ in $\stt^+$, which ends with rule $R$.
For the cases where $R$ is also a rule in $\st^+$, just apply $R$ on the proof-trees that we get from induction hypothesis for the sub-tree(s) of $\D$. If $R$ is $L \wedge$, we must use both $L \wedge_1$ and $L \wedge_2$, and $Lc$ in $\st^+$. If $R$ is $L \rightarrow$, we must use $Lw$ to prepare the context before using $L \rightarrow$ in $\st^+$.\\
($\Leftarrow$) Similar to the other direction, except that in the case for $Lc$ we must use Theorem \ref{thm:stt-lc-elim}.


\section{Proof of \ref{lem:l-nabla-dist}}\label{pr:l-nabla-dist}\quad \\
\begin{enumerate}
  \item \quad First, observe that

\begin{prooftree}
	\AXC{}
	\RightLabel{$Ta$}
	\UIC{$\Rightarrow \top$}

	\AXC{}
	\RightLabel{$(Id)$}
	\UIC{$C \Rightarrow C$}

	\RightLabel{$L \rightarrow$}
	\BIC{$\nabla (\top \rightarrow C) \Rightarrow C$}
\end{prooftree}


Now, let $\D$ be the proof-tree above with $C = \nabla A \vee \nabla B$.

\begin{prooftree}
	\AXC{}
	\RightLabel{$(Id)$}
	\UIC{$\nabla A \Rightarrow \nabla A$}
	\RightLabel{$R\vee_1$}
	\UIC{$\nabla A \Rightarrow \nabla A \vee \nabla B$}
	\RightLabel{$Lw$}
	\UIC{$\nabla A , \top \Rightarrow \nabla A \vee \nabla B$}
	\RightLabel{$R \rightarrow$}
	\UIC{$A \Rightarrow \top \rightarrow (\nabla A \vee \nabla B)$}

	\AXC{}
	\RightLabel{$(Id)$}
	\UIC{$\nabla B \Rightarrow \nabla B$}
	\RightLabel{$R\vee_2$}
	\UIC{$\nabla B \Rightarrow \nabla A \vee \nabla B$}
	\RightLabel{$Lw$}
	\UIC{$\nabla B , \top \Rightarrow \nabla A \vee \nabla B$}
	\RightLabel{$R \rightarrow$}
	\UIC{$B \Rightarrow \top \rightarrow (\nabla A \vee \nabla B)$}

	\RightLabel{$L\vee$}
	\BIC{$A \vee B \Rightarrow \top \rightarrow (\nabla A \vee \nabla B)$}
	\RightLabel{$N$}
	\UIC{$\nabla (A \vee B) \Rightarrow \nabla (\top \rightarrow (\nabla A \vee \nabla B))$}

	\AXC{$\D$}

	\RightLabel{$cut$}
	\BIC{$\nabla (A \vee B) \Rightarrow \nabla A \vee \nabla B$}
\end{prooftree}

Call this proof-tree $\D_1$. Construct $\D_n$ inductively as follows.

\begin{prooftree}
	\AXC{$\mathcal{D}_{n-1}$}
	\noLine
	\UIC{$\nabla^{n-1} (A \vee B) \Rightarrow \nabla^{n-1} A \vee \nabla^{n-1} B$}
	\RightLabel{$N$}
	\UIC{$\nabla^n (A \vee B) \Rightarrow \nabla (\nabla^{n-1} A \vee \nabla^{n-1} B)$}

	\AXC{$\mathcal{D}_1$}
	\noLine
	\UIC{$\nabla (\nabla^{n-1} A \vee \nabla^{n-1} B) \Rightarrow \nabla^n A \vee \nabla^n B$}
	
	\RightLabel{$cut$} \LeftLabel{$\mathcal{D}_n:$}
	\BIC{$\nabla^n (A \vee B) \Rightarrow \nabla^n A \vee \nabla^n B$}
\end{prooftree}
  \item \quad \begin{prooftree}
  \AXC{}
  \RightLabel{$(Id)$}
  \UIC{$A \Rightarrow A$}
  \RightLabel{$L \wedge_1$}
  \UIC{$A \wedge B \Rightarrow A$}
  \RightLabel{$N$} \doubleLine
  \UIC{$\nabla^n (A \wedge B) \Rightarrow \nabla^n A$}

  \AXC{}
  \RightLabel{$(Id)$}
  \UIC{$B \Rightarrow B$}
  \RightLabel{$L \wedge_2$}
  \UIC{$A \wedge B \Rightarrow B$}
  \RightLabel{$N$} \doubleLine	
  \UIC{$\nabla^n (A \wedge B) \Rightarrow \nabla^n B$}
  
  \RightLabel{$R \wedge$}
  \BIC{$\nabla^n (A \wedge B) \Rightarrow \nabla^n A \wedge \nabla^n B$}
\end{prooftree}
  \item \quad \begin{prooftree}
  \AXC{}
  \RightLabel{$(Id)$}
  \UIC{$A, \nabla (A \rightarrow B) \Rightarrow A$}

  \AXC{}
  \RightLabel{$(Id)$}
  \UIC{$A, \nabla (A \rightarrow B), B \Rightarrow B$}

  \RightLabel{$L \rightarrow$}
  \BIC{$\nabla (A \rightarrow B) , A \Rightarrow B$}
  \RightLabel{$N^{(n)}$} \doubleLine
  \UIC{$\nabla^{n+1} (A \rightarrow B) , \nabla^n A \Rightarrow \nabla^n B$}
  \RightLabel{$R \rightarrow$}
  \UIC{$\nabla^n (A \rightarrow B) \Rightarrow \nabla^n A \rightarrow \nabla^n B$}
\end{prooftree}
\end{enumerate}



\section{Proof of \ref{thm:sttp-eq-gsttp}}\label{pr:sttp-eq-gsttp}\quad \\
($\Rightarrow$) Easily follows from the fact that all rules of $\stt^+$ are just instances of $\gstt^+$'s rules.\\
($\Leftarrow$) We will use induction on the proof-tree $\D$ for $\Gamma \Rightarrow \Delta$ in $\gstt^+$, which ends with rule $R$.

The cases for Axioms are trivial.
For the rules that are common between two systems, just apply $R$ on the proof-tree(s) from the induction hypothesis.
In the cases for $L \wedge$, $L \vee$ and $L \rightarrow$, do the same, and also $\nabla cut$ the sequents proved in Lemma \ref{lem:l-nabla-dist} into the resulting sequent. Finally, if $R$ is $\nabla cut$, just apply $N$ $n$ times before using $cut$ in $\stt^+$.


\section{Proof of \ref{lem:gstt-top-redundant}}\label{pr:gstt-top-redundant}
Suppose $\D$ is a proof-tree for $\Gamma , \nabla^n \top \Rightarrow \Delta$ in $\gstt$ and consider different cases for $R$, the last rule in $\D$.
By induction on $\D$ we can assume that the theorem holds for its subtrees.
First, observe that the cases where $R$ is an axiom are trivial. In all other cases, just apply induction hypothesis on the subtrees of $\D$, and then apply $R$. Notice that $\nabla cut$ is not used except in the case where $R$ is $\nabla cut$ itself, where it is applied with a cut-formula of the same rank. So the resulting proof-tree will not have a higher rank than $\D$.



\section{Proof of \ref{thm:gstt-cut-reduction}}\label{pr:gstt-cut-reduction}

    We have two proof-trees
    \[
      \genfrac{}{}{0pt}{}{\D_0}{\Gamma \Rightarrow A}
      \hspace{3em}
      \genfrac{}{}{0pt}{}{\D_1}{\Sigma, \nabla^n A \Rightarrow \Delta}
    \]
    both of a lower rank than that of $A$, and we want to construct a proof-tree
    \[\genfrac{}{}{0pt}{}{\D}{\nabla^n \Gamma, \Sigma \Rightarrow \Delta} \]
    without increasing the cut rank.
  
    The construction takes place in different cases for the last rule that occurs in $\D_0$ and $\D_1$, which we call $R_0$ and $R_1$, respectively.

    The cases are divided into three parts. In the first part, our construction would depend only on $R_0$, and it would work in all cases for $R_1$. The second part is similar: Any of the remaining cases for $R_1$ will also cover all the cases for $R_0$. The cases that remain for the third part are the cases where our construction depends on \emph{both} $R_0$ and $R_1$, which are the cases that the cut-formula is altered in both of the proof-trees. In these cases, $R_1$ determines a specific form for the cut-formula, which will also determine $R_1$.
  
    In first two parts of the cases, we will need induction hypothesis only for premises of one of the subtrees. But in the third part, we need to use induction hypotesis for premises in both $\D_0$ and $\D_1$. So we will apply induction on $h(\D_0) + h(\D_1)$, generalizing all other parameters. Thus, the induction hypothesis reads like this: For any two proof-trees $\D_0'$ and $\D_1'$ such that $h(\D_0') + h(\D_1') < h(\D_0) + h(\D_1)$, where $\D_0'$ proves $\Gamma' \Rightarrow A'$ and $\D_1'$ proves $\Sigma', \nabla^{n'} A'\Rightarrow \Delta'$ for arbitrary $\Gamma'$, $\Sigma'$, $\Delta'$, $A'$ and $n'$, for which we have $\rho(\D_0'),\rho(\D_1') < \rho(A')$, the induction hypothesis gives us a prooftree, denoted by $\IH(\D_0', \D_1')$, which proves $\nabla^{n'}\Gamma', \Sigma' \Rightarrow \Delta'$, and we will also have $\rho(\IH(\D_0', \D_1')) < \rho(A')$.
  
    Now, in any of the following cases, suppose $\D_0$ and $\D_1$ end with any of the rules of $\gstt^+$, and construct the proof-tree $\D$ accordingly.

    \textbf{Part I.} First, assume that $R_0$ is an axiom and $R_1$ is any rule. The cases where $R$ is $Id$ or $L \bot$ would be trivial, and $R \top$ is handled by Lemma \ref{lem:gstt-top-redundant}.
    Now, let $R_0 \in \{ Lw, Rw, L \wedge, L \vee, L \rightarrow, \nabla cut, N \}$. In all these cases---again, independent of $\D_1$---it suffices to use induction on the premises(s) of this rule and $\D_1$ to remove the cut-formula from both subtrees. Then, we can apply $R$ again to get the desired sequent.\\
  
    \noindent($L \wedge$)

    Let $R$ be $L \wedge$, that is
    \begin{prooftree}
      \noLine
      \AXC{$\D_0'$}
      \UIC{$\Gamma, \nabla^r B, \nabla^r C \Rightarrow A$}
      
      \RightLabel{$L \wedge$}
      \UIC{$\Gamma, \nabla^r (B \wedge C) \Rightarrow A$}
   \end{prooftree}
   Then by applying $L \wedge$ on $\IH(\D_0', \D_1)$ we have
   \begin{prooftree}
    \noLine
    \AXC{$\D_0'$}
    \UIC{$\Gamma, \nabla^r B, \nabla^r C \Rightarrow A$}
    
    \noLine
    \AXC{$\D_1$}
    \UIC{$\Sigma, \nabla^n A\Rightarrow \Delta$}
    
    \RightLabel{IH}
    \BIC{$\nabla^n \Gamma, \nabla^{n+r} B, \nabla^{n+r} C, \Sigma \Rightarrow \Delta$}
  
    \RightLabel{$L \wedge$}
    \UIC{$\nabla^n \Gamma, \nabla^{n+r} (B \wedge C), \Sigma \Rightarrow \Delta$}
   \end{prooftree}\quad\\
  
   
  \noindent($L \vee$):
  Let
     \begin{prooftree}
       \noLine
       \AXC{$\D_0'$}
       \UIC{$\Gamma, \nabla^r B \Rightarrow A$}
       
       \noLine
       \AXC{$\D_0''$}
       \UIC{$\Gamma, \nabla^r C \Rightarrow A$}
       
       \RightLabel{$L \vee$}
       \BIC{$\Gamma, \nabla^r (B \vee C) \Rightarrow A$}
    \end{prooftree}
    Then
    \begin{prooftree}
      \noLine
      \AXC{$\D_0'$}
      \UIC{$\Gamma, \nabla^r B \Rightarrow A$}
      
      \noLine
      \AXC{$\D_1$}
      \UIC{$\Sigma, \nabla^n A \Rightarrow \Delta$}
      
      \RightLabel{IH}
      \BIC{$\nabla^n \Gamma, \nabla^{n+r} B, \Sigma \Rightarrow \Delta$}
      
  
      \noLine
      \AXC{$\D_0''$}
      \UIC{$\Gamma, \nabla^r C \Rightarrow A$}
      
      \noLine
      \AXC{$\D_1$}
      \UIC{$\Sigma, \nabla^n A \Rightarrow \Delta$}
      
      \RightLabel{IH}
      \BIC{$\nabla^n \Gamma, \nabla^{n+r} C, \Sigma \Rightarrow \Delta$}
  
      \RightLabel{$L \vee$}
      \BIC{$\nabla^n \Gamma, \nabla^{n+r} (B \vee C), \Sigma \Rightarrow \Delta$}
     \end{prooftree}\quad\\
  
   
  \noindent($L \rightarrow$):
  Let
   \begin{prooftree}
    \noLine
    \AXC{$\D_0'$}
    \UIC{$\Gamma, \nabla^{r+1} (B \rightarrow C) \Rightarrow \nabla^r B$}
    \noLine
    \AXC{$\D_0''$}
    \UIC{$\Gamma, \nabla^{r+1} (B \rightarrow C), \nabla^r C \Rightarrow A$}
    \RightLabel{$L \rightarrow$}
    \BIC{$\Gamma, \nabla^{r+1} (B \rightarrow C) \Rightarrow A$}
   \end{prooftree}
   Then
   \begin{prooftree}
    \noLine
    \AXC{$\D_0'$}
    \UIC{$\Gamma, \nabla^{r+1} (B \rightarrow C) \Rightarrow \nabla^r B$}
    \RightLabel{$N^n$} \doubleLine
    \UIC{$\nabla^n \Gamma, \nabla^{n+r+1} (B \rightarrow C) \Rightarrow \nabla^{n+r} B$}
    \RightLabel{$Lw$}
    \UIC{$\nabla^n \Gamma, \nabla^{n+r+1} (B \rightarrow C), \Sigma \Rightarrow \nabla^{n+r} B$}

    \noLine
    \AXC{$\D_0''$}
    \UIC{$\Gamma, \nabla^r C, \nabla^{r+1} (B \rightarrow C) \Rightarrow A$}
    \noLine
    \AXC{$\D_1$}
    \UIC{$\Sigma, \nabla^n A \Rightarrow \Delta$}
    \RightLabel{IH}
    \BIC{$\nabla^n \Gamma, \nabla^{n+r+1} (B \rightarrow C), \nabla^{n+r} C, \Sigma \Rightarrow \Delta$}

    \RightLabel{$L \rightarrow$}
    \BIC{$\nabla^n \Gamma, \nabla^{n+r+1} (B \rightarrow C), \Sigma \Rightarrow \Delta$}
   \end{prooftree}

  \noindent($\nabla cut$):
  Assume $\D_0$ ends with a $\nabla cut$ with cut-formula $A'$. Recall that by assumption $A'$ must have a lower rank than $A$.
   \begin{prooftree}
     \noLine
     \AXC{$\D_0'$}
     \UIC{$\Gamma \Rightarrow A'$}
     
     \noLine
     \AXC{$\D_0''$}
     \UIC{$\Pi, \nabla^{n'} A' \Rightarrow A$}
     
     \RightLabel{$\nabla cut$}
     \BIC{$\nabla^{n'} \Gamma, \Pi \Rightarrow A$}
   \end{prooftree}
   We must construct a proof-tree for $\nabla^{n + n'} \Gamma, \nabla^n \Pi, \Sigma \Rightarrow \Delta$. We can use the induction hypothesis first to remove $A$, and then use a low rank $\nabla cut$ to remove $A'$.
   \begin{prooftree}
     \noLine
     \AXC{$\D_0'$}
     \UIC{$\Gamma \Rightarrow A'$}
     
     \noLine
     \AXC{$\D_0''$}
     \UIC{$\Pi, \nabla^{n'} A' \Rightarrow A$}
  
     \noLine
     \AXC{$\D_1$}
     \UIC{$\Sigma, \nabla^n A \Rightarrow \Delta$}
  
     \RightLabel{IH}
     \BIC{$\nabla^n \Pi, \nabla^{n + n'} A', \Sigma \Rightarrow \Delta$}
     
  
     \RightLabel{$\nabla cut$}
     \BIC{$\nabla^{n + n'} \Gamma, \nabla^n \Pi, \Sigma \Rightarrow \Delta$}
   \end{prooftree}\quad\\  

  \noindent($N$):
  Let $A = \nabla B$ and suppose that $\D_0$ proves $\nabla \Gamma \Rightarrow \nabla B$ and $\D_1$ proves $\Sigma, \nabla^n (\nabla B) \Rightarrow \Delta$. Induction hypothesis for $\D_0$'s immediate sub-tree and $\D_1$ gives us $\nabla^{n+1} \Gamma, \Sigma \Rightarrow \Delta$.

  \textbf{Part II.} The rest of the cases for $R_0$ can't be solved independent of $R_1$. So in the second part, we will investigate the cases for $R_1$ where the solution can be constructed independent of $R_0$. Notice that this time we have less possibilities for $R_0$, because we have already solved most of them in the first part. We can assume that $R_0$ is either of $R\star$ for $\star \in \{\wedge, \vee_{1/2}, \rightarrow\}$.
  
   The cases where $R_1$ is an axiom are trivial. Notice that in $L \bot$ case, $\bot$ can't be the cut-formula, since all possible cases for $\D_0$ alter the right side of the sequent, but none of them are able to introduce $\bot$ on the right-side of a sequent.
   In the remaining cases, where the cut-formula is not principal in $R_1$, the construction is similar to the first part: Apply the same rule on the sequent from the induction hypothesis. But in cases where $A$ is principal in $R_1$, which are to be handeld in the third part, we must also use the induction hypothesis for $\D_0$, both with a different cut-fornula.
   
   We now address the second part of the cases. For the sake of briefness, we will only explain the cases for $L \wedge$, $R \vee_1$, $R \rightarrow$ and $N$, the last two of which are of special concern, in which we must use induction hypothesis with different $n$. The rest would be handled similarly. Now suppose $R_1$ is either of the following.\\
  
   
   \noindent($L \wedge$):
   Assume that $R_1$ is $L \wedge$, but the cut-formula is not principal.
   \begin{prooftree}
    \AXC{$\D_1'$} \noLine
    \UIC{$\Sigma, \nabla^n A, \nabla^r B, \nabla^r C \Rightarrow \Delta$}
    \RightLabel{$L \wedge$}
    \UIC{$\Sigma, \nabla^n A, \nabla^r (B \wedge C) \Rightarrow \Delta$}
   \end{prooftree}
   From induction hypothesis we have $\nabla^n \Gamma, \Sigma, \nabla^r B \Rightarrow \Delta$. By $L \wedge$ we have $\nabla^n \Gamma, \Sigma, \nabla^r (B \wedge C) \Rightarrow \Delta$.\\

   \noindent($R \vee_1$): Suppose that $R_1$ is $R \vee_1$.
   \begin{prooftree}
    \AXC{$\D_1'$} \noLine
    \UIC{$\Sigma, \nabla^n A \Rightarrow B$}
    \RightLabel{$R \vee_1$}
    \UIC{$\Sigma, \nabla^n A \Rightarrow B \vee C$}
   \end{prooftree}
   Again, use the induction hypothesis to get $\nabla^n \Gamma, \Sigma \Rightarrow B$, then apply $R \vee_1$ to reach the desired sequent.\\
  
  \noindent($R \rightarrow$): Assume $R_1$ is an instance of $R \rightarrow$.
  \begin{prooftree}
    \AXC{$\D_1'$} \noLine
    \UIC{$\nabla \Sigma, \nabla^{n+1} A, B \Rightarrow C$}
    \RightLabel{$R \rightarrow$}
    \UIC{$\Sigma, \nabla^n A \Rightarrow B \rightarrow C$}
   \end{prooftree}
  From induction hypothesis (with $n = n+1$), we have $\nabla^{n+1} \Gamma, \nabla \Sigma, B \Rightarrow C$. We can apply $R \rightarrow$ to get $\nabla^n \Gamma, \Sigma \Rightarrow B \rightarrow C$.\\
  
  \noindent($N$): Suppose $R_1$ is $N$.
  \begin{prooftree}
    \AXC{$\D_1'$} \noLine
    \UIC{$\Sigma, \nabla^n A \Rightarrow \Delta$}
    \RightLabel{$N$}
    \UIC{$\nabla \Sigma, \nabla^{n+1} A \Rightarrow \nabla \Delta$}
  \end{prooftree}
  If we assume that the cut-formula is $A$, from the induction hypothesis we have $\nabla^n \Gamma, \Sigma \Rightarrow \Delta$. By $N$ we have $\nabla^{n+1} \Gamma, \nabla \Sigma \Rightarrow \nabla \Delta$, which is the desired sequent. Notice that the cut-formula can't be $\nabla A$, because no candidate for $R_0$ proves a sequent with $\nabla$ on its right-side.
  
   \textbf{Part III.} In the last part of the proof, we will show how the construction takes place in the cases where the cut-formula is principal in $R_1$, which can be either of $L\star (\star \in \{\wedge, \vee, \rightarrow\})$.
   Any choice for $R_1$ also determines $R_0$, because in all candidates for $R_0$, which can be either of $R\star (\star \in \{\wedge, \vee_{1/2}, \rightarrow\})$, the cut-formula $A$ is principal.
   
   Now suppose $R_0$ and $R_1$ be instances of the following rules, respectively.\\

  \noindent($R \wedge$ and $L \wedge$): Suppose $R_0$ is $R \wedge$ and $R_1$ is $L \wedge$.
  \begin{prooftree}
    \noLine
    \AXC{$\D_0'$}
    \UIC{$\Gamma \Rightarrow A_1$}
    \noLine
    \AXC{$\D_0''$}
    \UIC{$\Gamma \Rightarrow A_2$}
    \RightLabel{$R \wedge$}
    \BIC{$\Gamma \Rightarrow A_1 \wedge A_2$}
    
    \noLine
    \AXC{$\D_1'$}
    \UIC{$\Sigma, \nabla^n A_1, \nabla^n A_2 \Rightarrow \Delta$}
    \RightLabel{$L \wedge$}
    \UIC{$\Sigma, \nabla^n (A_1 \wedge A_2) \Rightarrow \Delta$}
    
    \noLine
    \BIC{}
  \end{prooftree}
  Then
  \begin{prooftree}
    \AXC{$\D_0''$}\noLine
    \UIC{$\Gamma \Rightarrow A_2$}
    \AXC{$\D_0'$}\noLine
    \UIC{$\Gamma \Rightarrow A_1$}
    \AXC{$\D_1'$}\noLine
    \UIC{$\Sigma, \nabla^n A_1, \nabla^n A_2 \Rightarrow \Delta$}
    \RightLabel{$\nabla cut$}
    \BIC{$\nabla^n \Gamma, \Sigma, \nabla^n A_2 \Rightarrow \Delta$}
    \RightLabel{$\nabla cut$}
    \BIC{$(\nabla^n \Gamma)^2, \Sigma \Rightarrow \Delta$}
    \RightLabel{$(Lc)$}\doubleLine
    \UIC{$\nabla^n \Gamma, \Sigma \Rightarrow \Delta$}
  \end{prooftree}
  Notice that both instances of $\nabla cut$ in the above proof-tree have lower rank than $\D$. Also notice that we are using Theorem \ref{thm:stt-lc-elim}.\\
  
   \noindent($R \vee_1$ or $R \vee_2$, and $L \vee$):
   
  Suppose that $R_0$ is either of $R \vee_c ~ (c \in \{1,2\})$ and $R_1$ is $L \vee$.
  \begin{prooftree}
    \noLine
    \AXC{$\D_0'$}
    \UIC{$\Gamma \Rightarrow A_c$}
    \RightLabel{$R \vee_c$}
    \UIC{$\Gamma \Rightarrow A_1 \vee A_2$}
    \noLine
    \AXC{$\D_{10}$}
    \UIC{$\Sigma, \nabla^n A_1 \Rightarrow \Delta$}
    \noLine
    \AXC{$\D_{11}$}
    \UIC{$\Sigma, \nabla^n A_2 \Rightarrow \Delta$}
    \RightLabel{$L \vee$}
    \BIC{$\Sigma, \nabla^n (A_1 \vee A_2) \Rightarrow \Delta$}
    \noLine
    \BIC{}
  \end{prooftree}
  Then
  \begin{prooftree}
    \AXC{$\D_0'$}\noLine
    \UIC{$\Gamma \Rightarrow A_c$}
    \AXC{$\D_{1c}$}\noLine
    \UIC{$\Sigma, \nabla^n A_c \Rightarrow \Delta$}
    \RightLabel{$\nabla cut$}
    \BIC{$\nabla^n \Gamma, \Sigma \Rightarrow \Delta$}
  \end{prooftree}

  \noindent($\D_0$ and $L \rightarrow$): Let $R_0$ and $R_1$ be $R \rightarrow$ and $L \rightarrow$ respectively.
  \begin{prooftree}
    \noLine
    \AXC{$\D_0'$}
    \UIC{$\nabla \Gamma, A_1 \Rightarrow A_2$}
    \RightLabel{$R \rightarrow$}
    \UIC{$\Gamma \Rightarrow A_1 \rightarrow A_2$}
    \noLine
    \AXC{$\D_1'$}
    \UIC{$\Sigma, \nabla^{n+1} (A_1 \rightarrow A_2) \Rightarrow \nabla^n A_1$}
    \noLine
    \AXC{$\D_1''$}
    \UIC{$\Sigma, \nabla^{n+1} (A_1 \rightarrow A_2), \nabla^n A_2 \Rightarrow \Delta$}
    \RightLabel{$L \rightarrow$}
    \BIC{$\Sigma,  \nabla^{n+1} (A_1 \rightarrow A_2) \Rightarrow \Delta$}
    \noLine
    \BIC{}
  \end{prooftree}
  Let the following proof-tree be called $\D$.
  \begin{prooftree}
    \AXC{$\D_0'$} \noLine
    \UIC{$\nabla \Gamma, A_1 \Rightarrow A_2$}

    \AXC{$\D_0$} \noLine
    \UIC{$\Gamma \Rightarrow A_1 \rightarrow A_2$}
    \AXC{$\D_1''$} \noLine
    \UIC{$\Sigma, \nabla^{n+1} (A_1 \rightarrow A_2), \nabla^n A_2 \Rightarrow \Delta$}
    \RightLabel{IH}
    \BIC{$\nabla^{n+1} \Gamma, \Sigma, \nabla^n A_2 \Rightarrow \Delta$}
    \LeftLabel{$\D$} \RightLabel{$\nabla cut$}
    \BIC{$(\nabla^{n+1} \Gamma)^2, \nabla^n A_1, \Sigma \Rightarrow \Delta$}
  \end{prooftree}
  Then
  \begin{prooftree}
    \AXC{$\D_0$} \noLine
    \UIC{$\Gamma \Rightarrow A_1 \rightarrow A_2$}
    \AXC{$\D_1'$} \noLine
    \UIC{$\Sigma, \nabla^{n+1} (A_1 \rightarrow A_2) \Rightarrow \nabla^n A_1$}
    \RightLabel{IH}
    \BIC{$\nabla^{n+1} \Gamma, \Sigma \Rightarrow \nabla^n A_1$}

    \AXC{$\D$}

    \RightLabel{$\nabla cut$}
    \BIC{$(\nabla^{n+1} \Gamma)^3, \Sigma^2 \Rightarrow \Delta$}

    \RightLabel{$(Lc)$} \doubleLine
    \UIC{$\nabla^{n+1} \Gamma, \Sigma \Rightarrow \Delta$}
  \end{prooftree}

  \vspace{5mm}

  Now we have a construction for any two possible pair of rules, in $\gstt^+$. This concludes the proof of the theorem in all cases.
  \qed
  




\section{Proof of \ref{thm:gstt-eq-gsttp}}\label{pr:gstt-eq-gsttp}

First, we will show that for any non-zero-rank proof of $\Gamma \Rightarrow \Delta$ like $\D$ in $\gstt^+$, there is another proof of the same sequent with a strictly lower rank. Suppose $\D$ has subtree(s) called $\mathcal{D}_0$ (and possibly $\mathcal{D}_1$, if the last rule has two premises). Using induction on $h(\D)$, the induction hypothesis for $\D_i ~(i \in \{0,1\})$ gives us a proof-tree with the same conclusion, which we call $\IH(\D_i)$, but with a lower rank, i.e. $\rho(\IH(\D_i)) < \rho(\D_i)$. We now consider two cases for the last rule in $\D$.

\begin{enumerate}[label=\Roman*]
	\item If the last rule of $\D$ is of a lower rank than $\D$ itself, which means that the last rule in $\D$ is not the $\nabla cut$ instance with the maximum rank, then we can apply the same last rule on $\IH(\D_0)$ (and possibly $\D_1$), to construct a proof of $\Gamma \Rightarrow \Delta$ with a strictly lower rank.
	
	\item If the last rule of $\D$ is the instance of $\nabla cut$ rule instance with the maximum rank, we can apply Theorem \ref{thm:gstt-cut-reduction} to $\IH(\D_0)$ and $\IH(\D_1)$ to prove the same sequent as it would be proved by $\nabla cut$, but with a strictly lower rank.
\end{enumerate}
So for any proof of $\Gamma \Rightarrow \Delta$ in $\gstt^+$, we also have a proof of rank $0$, which is cut-free, and hence provable in $\gstt$.



\section{Proof of \ref{thm:gstt-eq-stp}}\label{pr:gstt-eq-stp}
Trivial.



\pagebreak

\part*{\emph{Notes}}

\begin{itemize}
  \item We can't eliminate weakening rules from $\stt$, because of $N$.

  \item $L \wedge$ in $\stt$ gives us (partial) normality.
  
  \item We need Inversion Lemmas and Contraction Admissibility Theorem to preserve length, because we will use induction more than one on the case for $L \wedge$ in Contraction Admissibility Theorem.
  
  \item We need $Lw$ in $\stt$ to add more than one formula to the left-side, at once, since we will have to do so in $L \wedge$ Inversion Lemma, in the case where the $A \wedge B$ is principal in $Lw$.
  
  \item We don't need to prove inverse of $L \rightarrow$.

  \item Inversion Lemmas for the left-rules are generalized with arbitrary number of $\nabla$s, because in the case for $R \rightarrow$ we would have a $\nabla$ added to the left-side of the premise.
  
\end{itemize}

\end{document}
